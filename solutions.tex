\documentclass{article}
\usepackage{graphicx} % Required for inserting images
\usepackage[english]{babel}
\usepackage{amssymb}
\usepackage{amsthm}
\usepackage{enumitem} 
\usepackage{amsmath}
\usepackage{amsfonts}
\usepackage{tikz}
\usetikzlibrary{matrix}
\usepackage[english]{babel}
\usepackage{mathtools}
\usepackage[a4paper, total={6in, 8in}]{geometry}
\usepackage{cite}
\usepackage{graphicx}
\graphicspath{ {./images/} }


\newtheorem{theorem}{Theorem}[section]
\newtheorem{definition}[theorem]{Definition}
\newtheorem{lemma}[theorem]{Lemma}
\newtheorem{proposition}[theorem]{Proposition}
\newtheorem{corollary}[theorem]{Corollary}
\newtheorem{example}[theorem]{Example}
\newtheorem{remark}[theorem]{Remark}
\newtheorem{exercise}[theorem]{Exercise}


\title{Hatcher solutions}
\author{Saxon Supple}
\date{July 2025}

\begin{document}

\maketitle
\begin{exercise}
Construct an explicit deformation retraction of the torus with one point deleted onto a graph consisting of two circles intersecting in a point, namely, longitude and meridian circles of the torus.
\end{exercise}
\begin{proof}
We can model the torus as the disk $D^2:=\{(x,y)\in\mathbb{R}^2:x^2+y^2\leq 1\}$ where we split the boundary into $4$ equal arcs and identify opposite arcs under the quotient map $q:D^2\to T$. Let $A:=q(\partial D^2)$ be the subspace we wish to retract $T$ onto. Without loss of generality, suppose that we remove the point $(0,0)$. Given a point $x$, we want to send it to $\frac{x}{\|x\|}$ along the line segment from $x$ to $\frac{x}{\|x\|}$. We thus define the homotopy \[F:T\times I\to T:(q(x),t)\mapsto q\left(x(1-t)+\frac{x}{\|x\|}t\right).\] This homotopes between the identity map and a retraction of $T$ onto $A$ while fixing all points of $A$ throughout, and hence is a deformation retraction.
\end{proof}

\begin{exercise}
Construct an explicit deformation retraction of $\mathbb{R}^n\setminus\{0\}$ onto $S^{n-1}$.
\end{exercise}
\begin{proof}
We define a homotopy \[F:\mathbb{R}^n\setminus\{0\}\times I\to S^{n-1}:(x,t)\mapsto x(1-t)+\frac{x}{\|x\|}t.\] This homotopes between the identity map on $\mathbb{R}^n\setminus\{0\}$ and a retraction of $\mathbb{R}^n\setminus\{0\}$ onto $S^{n-1}$, and hence is a deformation retraction.
\end{proof}

\begin{exercise}
\begin{enumerate}
\item[(a)] Show that the composition of homotopy equivalences $X\to Y$ and $Y\to Z$ is a homotopy equivalence $X\to Z$. Deduce that homotopy equivalence is an equivalence relation.
\item[(b)] Show that the relation of homotopy among maps $X\to Y$ is an equivalence relation.
\item[(c)] Show that a map homotopic to a homotopy equivalence is a homotopy equivalence.
\end{enumerate}
\end{exercise}
\begin{proof}
\begin{enumerate}
\item[(a)] Let $\phi:X\to Y$ and $\psi: Y\to Z$ be homotopy equivalences. There then exist homotopy inverses $\phi_1:Y\to X$ and $\psi_1: Z\to Y$ of $\phi$ and $\psi$ respectively.\[(\phi_1\circ\psi_1)\circ(\psi\circ\phi)= \phi_1\circ(\psi_1\circ\psi)\circ\phi\simeq\phi_1\circ\phi\simeq\text{id}_X\] and\[(\psi\circ\phi)\circ(\phi_1\circ\psi_1)=\psi\circ(\phi\circ\phi_1)\circ\psi_1\simeq\psi\circ\psi_1\simeq\text{id}_Z\]so $\psi\circ\phi$ is a homotopy equivalence.
\item[(b)] Let $\phi,\psi,\rho$ be continuous maps from $X$ to $Y$. Clearly $\phi\simeq\phi$. Furthermore, if $F:X\times I\to Y$ is a homotopy from $\phi$ to $\psi$, then $G:X\times I\to Y:(x,t)\mapsto F(x,1-t)$ is a homotopy from $\psi$ to $\phi$. Finally, let $A:X\times I\to Y$ be a homotopy from $\phi$ to $\psi$ and let $B:X\times I\to Y$ be a homotopy from $\psi$ to $\rho$. We define a map $C:X\times I\to Y$ by \[C(x,t)=\begin{cases}
    A(x,2t)\text{ if }0\leq t\leq \frac{1}{2}\\B(x,2t-1)\text{ if }\frac{1}{2}<t\leq 1.
\end{cases}\]$X\times[0,\frac{1}{2}]$ and $X\times[\frac{1}{2},1]$ are both closed subsets of $X\times I$ with $X\times[0,\frac{1}{2}]\cup X\times[\frac{1}{2},1]=X\times I$. Furthermore, $C_{|{X\times[0,\frac{1}{2}]}}$ and $C_{|{X\times[\frac{1}{2},1]}}$ are both continuous with respect to the induced topologies on $X\times[0,\frac{1}{2}]$ and $X\times[\frac{1}{2},1]$ respectively. Hence, $C$ is continuous on $X\times I$ and is thus a homotopy from $\phi$ to $\rho$.
\item[(c)] Let $\phi:X\to Y$ be a homotopy equivalence with homotopy inverse $\psi:Y\to X$. Let $\phi_1:X\to Y$ be homotopic to $\phi$. Then \[\phi_1\circ\psi\simeq\phi\circ\psi\simeq\text{id}_Y\] and \[\psi\circ\phi_1\simeq\psi\circ\phi\simeq\text{id}_X.\] Hence $\phi_1$ is a homotopy equivalence.
\end{enumerate}
\end{proof}

\begin{exercise}
A $\textbf{deformation retraction in the weak sense}$ of a space $X$ to a subspace $A$ is a homotopy $f_t:X\to X$ such that $f_0=\mathbb{1}$, $f_1(X)\subset A$, and $f_t(A)\subset A$ for all $t$. Show that if $X$ deformation retracts to $A$ in this weak sense, then the inclusion $A\hookrightarrow X$ is a homotopy equivalence.
\end{exercise}
\begin{proof}
Let $\iota:A\to X$ denote the inclusion $A\hookrightarrow X$. Define $\phi:X\to A:x\mapsto f_1(x)$, which is well-defined since $f_1(X)\subset A$. Then \[\phi\circ\iota={f_1}_{|A}\simeq {f_0}_{|A}=\text{id}_A.\]This is a valid homotopy of maps $A\to A$ since $f_1(A)\subset A\forall t$. Furthermore,\[\iota\circ\phi=f_1\simeq\text{id}_X.\] Hence $\iota$ is a homotopy equivalence with homotopy inverse $\phi$.
\end{proof}

\begin{exercise}
Show that if a space $X$ deformation retracts to a point $x\in X$, then for each neighbourhood $U$ of $x$ in $X$ there exists a neighbourhood $V\subset U$ of $x$ such that the inclusion map $V\hookrightarrow U$ is nullhomotopic.
\end{exercise}
\begin{proof}
Without loss of generality, let $U$ be an open neighbourhood of $x$. Define \[F:X\times I\to X:(y,t)\mapsto f_t(y)\] and let $A:=F^{-1}(U)$. $A$ is an open set containing the slice $\{x\}\times I$, so by the Tube Lemma there exists a tube $V\times I\subset X\times I$, where $V$ is open and contains $x$. Furthermore, $V\times I\subset A$, so $V$ consists only of points in $U$ (since $V\times\{0\}\subset U\times\{0\}$) which stay in $U$ throughout the homotopy $f_t$. Let $\iota:V\hookrightarrow U$ be the inclusion map $V\hookrightarrow U$. Then $f_t:X\to X$ restricts to a homotopy ${f_t}_{|V}:V\to U$ and hence $\iota\simeq {f_1}_{|V}$, where ${f_1}_{|V}$ is the constant map, so $\iota$ is nullhomotopic.
\end{proof}

\begin{exercise}
\begin{enumerate}
\item[(a)] Let $X$ be the subspace of $\mathbb{R}^2$ consisting of the horizontal segment $[0,1]\times\{0\}$ together with all the vertical segments $\{r\}\times[0,1-r]$ for $r$ a rational number in $[0,1]$. Show that $X$ deformation retracts to any point in the segment $[0,1]\times\{0\}$, but not to any other point.
\item[(b)] Let $Y$ be the subspace of $\mathbb{R}^2$ that is the union of an infinite number of copies of $X$ arranged as in the figure below. Show that $Y$ is contractible but does not deformation retract onto any point.
\includegraphics[scale=0.5]{Screenshot 2025-07-20 at 19-46-35 AT.dvi - AT.pdf.png}
\item[(c)] Let $Z$ be the zigzag subspace of $Y$ homeomorphic to $\mathbb{R}$ indicated by the heavier line. Show there is a deformation retraction in the weak sense of $Y$ onto $Z$, but no true deformation retraction.
\end{enumerate}
\end{exercise}
\begin{proof}
\begin{enumerate}
\item[(a)] We define a homotopy \[f_t:X\to X:(a,b) \mapsto (a,b(1-t)).\] $f_t$ fixes $[0,1]\times\{0\}$, $f_0=\text{id}_X$ and $f_1$ is a retraction from $X$ to $[0,1]\times\{0\}$, so $f_t$ is a deformation retraction. Hence $[0,1]\times\{0\}$ is a deformation retract of $X$. Any point in $[0,1]\times\{0\}$ is then clearly a deformation retract of $[0,1]\times\{0\}$, so  $X$ deformation retracts to any point in $[0,1]\times\{0\}$ by transitivity. Now suppose $X$ deformation retracts to some other point $x\in X$. We can then find a neighbourhood $U$ of $x$ of the form $B_\epsilon(x)\cap X$ for some $\epsilon > 0$, which does not contain $[0,1]\times\{0\}$. However, $U$ is not path-connected, so there does not exist a neighbourhood $V\subset U$ of $x$ such that the inclusion map $V\hookrightarrow U$ is nullhomotopic; a contradiction.
\item[(b)] Define a homotopy $f_t:Y\to Y$ where $f_t(y)$ sends $y$ a distance of $t$ to the right along $Y$. Then $f_0=\text{id}_Y$ and $f_1(Y)=Z$, where $Z$ is the zigzag subspace of $Y$. $Z$ is contractible, being homeomorphic to $\mathbb{R}$, and hence there exists a homotopy $h_t:Z\to Z$ from $\text{id}_Z$ to a constant function. We can then define a homotopy $F_t:Y\to Y$ of $\text{id}_Y$ to a constant map by\[F_t(y)=\begin{cases}
    f_{2t}(y)\text{ if }0\leq t\leq\frac{1}{2},
    \\h_{2t-1}\circ f_1(y)\text{ if }\frac{1}{2}< t\leq 1.
\end{cases}\]

Now suppose $Y$ deformation retracts onto a point $y$. We can then find a sufficiently small open neighbourhood $U$ of $y$ such that for every neighbourhood $V\subset U$ of $y$, $U$ is not path-connected, and hence the inclusion map $V\hookrightarrow U$ is not nullhomotopic; a contradiction. Hence $Y$ does not deformation retract onto a point $y$.
\item[(c)] Since $Z$ deformation retracts onto a point, if there were a deformation retract of $Y$ onto $Z$, then there would be a deformation retract of $Y$ onto a point. However, (b) showed that this is not the case, and hence $Y$ does not deformation retract onto $Z$. However, the homotopy $f_t$ defined in (b) is a weak deformation retract of $Y$ onto $Z$.
\end{enumerate}
\end{proof}

\begin{exercise}
Fill in the details in the following construction from
[Edwards 1999] of a compact space $Y\subset\mathbb{R}^3$ with the
same properties as the space $Y$ in Exercise $6$, that is, $Y$
is contractible but does not deformation retract to any
point. To begin, let $X$ be the union of an infinite sequence of cones on the Cantor set arranged end-to-end,
as in the figure. Next, form the one-point compactification of $X\times\mathbb{R}$. This embeds in $\mathbb{R}^3$ as a closed disk with curved ‘fins’ attached along circular arcs, and with the one-point compactification of $X$ as a cross-sectional slice.
The desired space $Y$ is then obtained from this subspace of $\mathbb{R}^3$ by wrapping one more
cone on the Cantor set around the boundary of the disk.

\includegraphics[scale=0.5]{Screenshot 2025-07-21 at 17-47-12 AT.dvi - AT.pdf.png}
\end{exercise}

\begin{exercise}
For $n>2$, construct an $n$-room analog of the house with two rooms.
\end{exercise}

\begin{exercise}
Show that a retract of a contractible space is contractible.
\end{exercise}
\begin{proof}
Let $X$ be a contractible space with $A$ a retract of $X$. Let $f_t:X\to X$ be a homotopy from $\text{id}_X$ to a constant map, and let $r:X\to A$ be a retraction of $X$ onto $A$. $r\circ {f_0}_{|A}=\text{id}_A$ and $r\circ {f_1}_{|A}$ is constant due to ${f_1}_{|A}$ being constant. Hence $\text{id}_A$ is homotopic to a constant map so $A$ is contractible.
\end{proof}

\begin{exercise}
Show that a space $X$ is contractible iff every map $f:X\to Y$, for arbitrary $Y$, is
nullhomotopic. Similarly, show $X$ is contractible iff every map $f:Y\to X$ is nullhomotopic.
\end{exercise}
\begin{proof}
$(\impliedby)$: Let $Y=X$ and $f=\text{id}_X$. Then $\text{id}_X$ is nullhomotopic so $X$ is contractible.

$(\implies)$: Let $g_t:X\to X$ be a homotopy from $\text{id}_X$ to a constant map. Then define a homotopy $h_t:X\to Y$ by $h_t(x)=f\circ g_t(x)$. This is a homotopy between $f$ and a constant map, so $f$ is nullhomotopic.

$(\impliedby)$: Again, let $Y=X$ and $f=\text{id}_X$.

$(\implies)$: $g_t\circ f$ is a homotopy from $f$ to a constant map so $f$ is nullhomotopic.
\end{proof}

\begin{exercise}
Show that $f:X\to Y$ is a homotopy equivalence if there exist maps $g,h:Y\to X$ such that $fg\simeq\mathbb{1}$ and $hf\simeq\mathbb{1}$. More generally, show that $f$ is a homotopy equivalence if $fg$ and $hf$ are homotopy equivalences.
\end{exercise}
\begin{proof}
Let $\phi=h\circ f\circ g:Y\to X$. Then\[f\circ\phi=f\circ(h\circ f\circ g)=f\circ(h\circ f)\circ g\simeq f\circ g\simeq\mathbb{1}\] and\[\phi\circ f=(h\circ f\circ g)\circ f=h\circ(f\circ g)\circ f\simeq h\circ f\simeq\mathbb{1}\] so $\phi$ is a homotopy inverse of $f$ so $f$ is a homotopy equivalence.

Let $g_1$ and $h_1$ be the homotopy inverses of $fg$ and $hf$ respectively. We then have that $f(gg_1)\simeq\mathbb{1}$ and $(h_1h)f\simeq\mathbb{1}$, and hence $f$ is a homotopy equivalence with homotopy inverse $h_1hfgg_1$.
\end{proof}

\begin{exercise}
Show that a homotopy equivalence $f:X\to Y$ induces a bijection between the set of path-components of $X$ and the set of path-components of $Y$, and that $f$ restricts to
a homotopy equivalence from each path-component of $X$ to the corresponding path-component of $Y$. Prove also the corresponding statements with components instead
of path-components. Deduce that if the components of a space $X$ coincide with its
path-components, then the same holds for any space $Y$ homotopy equivalent to $X$.
\end{exercise}
\begin{proof}
Let $\sim_X$ be the equivalence relation on $X$ given by $a\sim_X b$ if and only if $a$ and $b$ are in the same path-component. Similarly define the relation $\sim_Y$ on $Y$. Let $g:Y\to X$ be a homotopy inverse of $f$. We let $\phi_t$ and $\psi_t$ be homotopies from $\text{id}_Y$ to $f\circ g$ and from $\text{id}_X$ to $g\circ f$ respectively.


\textbf{We first show that $f$ induces a well-defined map on path components.}\\Let $x_0\sim_X x_1$. Let $\phi:I\to X$ be a path from $x_0$ to $x_1$. Then $f\circ\phi:I\to Y$ is a path from $f(x_0)$ to $f(x_1)$ so $f(x_0)\sim_Y f(x_1)$.

\textbf{We now show that $f$ induces a surjective map on path components.}\\Let $y\in Y$. Then $y=\phi_0(y)\sim_Y \phi_1(y)=f(g(y))$, and hence $y$ is in the same path component as a point in the image of $f$.

\textbf{We finally show that $f$ induces an injective map on path components.}\\ Suppose that $f(x_0)\sim_Yf(x_1)$. Then $g\circ f(x_0)\sim_X g\circ f(x_1)$ and hence $\psi_1(x_0)\sim_X\psi_1(x_1)$. We also have that $x_0=\psi_0(x_0)\sim_X\psi_1(x_0)$ and $x_1=\psi_0(x_1)\sim_X\psi_1(x_1)$, and so by transitivity\[x_0\sim_X\psi_1(x_0)\sim_X\psi_1(x_1)\sim_X x_1\implies x_0\sim_X x_1.\]Hence $f$ induces a bijection on path-components.\newline

Let $[x]_X$ be the path component of some point $x\in X$ and let $[f(x)]_Y$ be the path component in $Y$ of $f(x)$. $f$ then restricts to a map \[f_{|[x]_X}:[x]_X\to [f(x)]_Y.\] Furthermore, $g\circ f(x)\sim_X x$, and hence $g$ also restricts to a map \[g_{|[f(x)]_Y}:[f(x)]_Y\to[x]_X.\] We can then also restrict the homotopies $\phi_t$ and $\psi_t$ to \[\phi_{t|[f(x)]_Y}:[f(x)]_Y\to[f(x)]_Y\] and \[\psi_{t|[x]_X}:[x]_X\to[x]_X\] respectively, since $\forall x_0\in[x]_X$ we have $\psi_t(x_0)\sim_Xx_0\forall t$, and $\forall y_0\in[f(x)]_Y$ we have $\phi_t(y_0)\sim_Y y_0\forall t$. Hence $f_{|[x]_X}$ is a homotopy equivalence between $[x]_X$ and $[f(x)]_Y$.

Now let $\sim_X$ and $\sim_Y$ be the corresponding relations for connected components.

\textbf{We first show that $f$ induces a well-defined map on connected components.}
Let $x_0\sim_X x_1$. Let $Z$ be a connected subspace containing $x_0$ and $x_1$. The image of a connected space is connected, and hence $f(x_0)\sim_Y f(x_1)$, as both are contained in $f(Z)$.

\textbf{We next show that $f$ induces a surjective map on connected components.}
Let $y\in Y$. Let $[y]_Y$ be the connected component of $y$. As show earlier, $y$ is in the same path component as $f\circ g(y)$, and hence is in the same connected component as $f\circ g(y)$. Hence, $[y]_Y$ is in the image of the map on connected components induced by $f$.

\textbf{We finally show that $f$ induces an injective map on connected components.}
Let $f(x_0)\sim_Yf(x_1)$. Then $g\circ f(x_0)\sim_X g\circ f(x_1)$. We then have that $x_0$ is in the same path-component as $g\circ f(x_0)$ and $x_1$ is in the same path-component as $g\circ f(x_1)$, and hence that $x_0$ is in the same connected component as $g\circ f(x_0)$ and $x_1$ is in the same connected component as $g\circ f(x_1)$. Hence by transitivity\[x_0\sim_X g\circ f(x_0)\sim_Xg\circ f(x_1)\sim_X x_1\implies x_0\sim_X x_1.\] Hence $f$ induces a bijection on connected components.

Let $[x]_X$ be the connected component of some point $x\in X$ and let $[f(x)]_Y$ be the connected component of $f(x)$ in $Y$. $f$ then restricts to a map\[f_{|[x]_X}:[x]_X\to [f(x)]_Y.\]Furthermore, $g\circ f(x)\sim_X x$, and hence $g$ also restricts to a map \[g_{|[f(x)]_Y}:[f(x)]_Y\to[x]_X.\]We can then also restrict the homotopies $\phi_t$ and $\psi_t$ to \[\phi_{t|[f(x)]_Y}:[f(x)]_Y\to[f(x)]_Y\] and \[\psi_{t|[x]_X}:[x]_X\to[x]_X\] respectively, since $\forall x_0\in[x]_X$ we have $\psi_t(x_0)\sim_Xx_0\forall t$, and $\forall y_0\in[f(x)]_Y$ we have $\phi_t(y_0)\sim_Y y_0\forall t$. Hence $f_{|[x]_X}$ is a homotopy equivalence between $[x]_X$ and $[f(x)]_Y$.

Finally, suppose that the connected components of a space $X$ coincide with its path-components, and let $Y$ be homotopy equivalent to $X$. Let $f:X\to Y$ be a homotopy equivalence. Let \[f_\#:\{\text{Path/connected components of } X\}\to\{\text{Path components of } Y\}\] and \[f_*:\{\text{Path/connected components of } X\}\to\{\text{Connected components of } Y\}\] be the maps induced by $f$. We then have a bijection between path-components of $Y$ and connected components of $Y$ given by $f_*\circ f_\#^{-1}$. Furthermore, given any path-component $A\subseteq Y$, $f_*\circ f_\#^{-1}(A)$ is the connected component containing $A$. Hence there must be at most one path-component per connected component, since otherwise multiple path-components would be mapped to the same connected component by $f_*\circ f_\#^{-1}$, and it would cease to be injective. Hence the connected components of $Y$ coincide with its path-components.
\end{proof}

\begin{exercise}
Show that any two deformation retractions $r_t^0$ and $r_t^1$ of a space $X$ onto a subspace $A$ can be joined by a continuous family of deformation retractions $r_t^s$, $0\leq s\leq 1$, of $X$ onto $A$, where continuity means that the map $X\times I\times I\to X$ sending $(x,s,t)$ to $r_t^s(x)$ is continuous.
\end{exercise}
\begin{proof}
We know what $r_t^s$ is on $X\times\{0,1\}\times I\cup A\times I\times I\cup X\times I\times\{0\}$.

\noindent Define $r_t^s$ by\[r_t^s(x)=\begin{cases}
    r_t^0\circ r_{2ts}^1(x)\text{ if } 0\leq s <\frac{1}{2},\\r_{2t(1-s)}^0\circ r_t^1(x)\text{ if }\frac{1}{2}\leq s\leq 1.
\end{cases}\] If $s=0$, then $r_t^0=r_t^0\circ r_0^1=r_t^0$.

\noindent If $s=1$, then $r_t^1=r_0^0\circ r_t^1=r_t^1$.

\noindent If $a\in A$, then $r_t^s(a)=a\forall s,t$.

\noindent If $t=0$, then $r_0^s=\text{id}_X\forall s$.

\noindent If $t=1$, then $r_1^0(x)\in A\forall x$ and $r_1^1(x)\in A\forall x$, and hence $r_1^s(x)\in A\forall x$.
\end{proof}

\begin{exercise}
Given positive integers $v$, $e$ and $f$ satisfying $v-e+f=2$, construct a cell structure on $S^2$ having $v$ $0$-cells, $e$ $1$-cells, and $f$ $2$-cells.
\end{exercise}
\begin{proof}
For $(v,e,f)=(2+m,m+n+1,n+1)$ with $m>0,n>1$, we can use the following construction: Begin with $0$-cells $e_1^0,...,e_{2+m}^0$. Then attach $n-2$ loops $e_1^1,...,e_{n-2}^1$ to $e_1^0$, and add two $1$-cells $e_{n-1}^1$ and $e_n^1$, each with one endpoint attached to $e_1^0$ and the other endpoint attached to $e_{2+m}^0$. Then add $1$-cells $e_{n+1}^1,...,e_{n+1+m}^1$, where $e_{n+i}^1$ is attached to $e_i^0$ and $e_{i+1}^0$. Then add $2$-cells $e_1^2,...,e_{n+1}^2$, where the first $n-2$ are attached to the $n-2$ loops, the next two are attached to $e_1^0\cup...\cup e_{2+m}^0\cup e_{n-1}^1\cup e_{n+1}^1\cup...e_{n+1+m}^1$ and $e_1^0\cup...\cup e_{2+m}^0\cup e_n^1\cup e_{n+1}^1\cup...e_{n+1+m}^1$ respectively. Finally, attach the last $2$-cell to $e_1^0\cup e_2^0\cup e_1^1\cup...\cup e_n^1$.

\includegraphics[scale=0.5]{Screenshot (1340).png}

If $v=2$, then $e=f$. First suppose $e>1$. We can then form $S^2$ by starting with $e_1^0$ and $e_2^0$, and then attaching $e-2$ loops to $e_1^0$ and two $1$-cells each connecting the two $0$-cells, and finally attaching $e$ $2$-cells.

\includegraphics[scale=0.5]{Screenshot (1341).png}

The case $(v,e,f)=(2,1,1)$ is then just the standard construction of starting with two points, attaching two lines joining them to form a circle, filling in the centre of the circle, and then identifying the two lines.

Now consider the case $(v,e,f)=(2+m,m+2,2)$. That is, the case where $n=1$. The following diagram works:

\includegraphics[scale=0.5]{Screenshot (1342).png}

For the case $(v,e,f)=(2+m,m+1,1)$, which corresponds to $n=0$, the following diagram works:

\includegraphics[scale=0.5]{Screenshot (1343).png}

For the case $(v,e,f)=(1,x,x+1)$, the following diagram works:

\includegraphics[scale=0.5]{Screenshot (1344).png}
\end{proof}

\begin{exercise}
Enumerate all the subcomplexes of $S^\infty$, with the cell structure on $S^\infty$ that has $S^n$ as its $n$-skeleton.
\end{exercise}
\begin{proof}
$X^n=S^n$ is a subcomplex of $S^\infty$ for every $n$. Also, for each $X^n=S^n$, $X^n\cup e_{1}^{n+1}$ and $X^n\cup e_{2}^{n+1}$ are subcomplexes. It is not possible to have a subcomplex formed by a proper subset of $X^n$ and some $\emptyset\neq\phi\subseteq X^{n+i}\setminus X^n$ for some $i\geq 1$, since $\phi$ attaches to the whole of $X^n$, and hence the subcomplex, being closed, would contain $X^n$; a contradiction. Hence, if a subcomplex contains an $i$-cell for $i>0$, it must also contain $X^{i-1}$. Hence the only possible subcomplexes are:
\[\emptyset, S^\infty, e_1^0,e_2^0, S^{i-1}\cup e_1^i,S^{i-1}\cup e_2^i \text{ for } i>0,\text{ and } S^n\text{ for }n\geq 0.\]
\end{proof}

\begin{exercise}
Show that $S^\infty$ is contractible.
\end{exercise}
\begin{proof}
For $k>0$, each $k$-skeleton $X^k\subseteq S^\infty$ is $S^{k-1}\cup e_1^k\cup e_2^k$, where $S^{k-1}\cup e_1^k$ and $S^{k-1}\cup e_2^k$ are both the disk $D^k$. Furthermore, for $k>1$, each $S^{k-1}\cup e_i^k$ can be deformation retracted into either $X^{k-2}\cup e_1^{k-1}$ or $X^{k-2}\cup e_2^{k-1}$. We can then repeat this recursively until we reach $X^0\cup e_i^{1}$, which we can then deformation retract to a single point in $X^0$. For $k>1$, let \[F_k:S^{k-1}\cup e_1^k\times I\to S^{k-1}\cup e_1^k\] be a deformation retraction of $S^{k-1}\cup e_1^k$ onto $X^{k-2}\cup e_1^{k-1}$, and for $k=1$, let \[F_1:X^0\cup e_1^1\times I\to X^0\cup e_1^1\] be a deformation retraction of $X^0\cup e_1^1$ onto $e_1^0$. Then, given an $x\in S^\infty$, let $n$ be the smallest positive integer such that $x\in X^{n-1}\cup e_1^n$. We then define a homotopy $h_t$ between $\text{id}_{S^\infty}$ and a constant map by\[h_t(x)=\begin{cases}
    x&\text{ if }0\leq t<\frac{1}{2^{n+1}},\\F_n(x,2^n(t-\frac{1}{2^{n+1}}))&\text{ if }\frac{1}{2^{n+1}}\leq t<\frac{1}{2^n},\\F_{n-1}(F_n(x,1),2^{n-1}(t-\frac{1}{2^{n}}))&\text{ if }\frac{1}{2^n}\leq t<\frac{1}{2^{n-1}},\\\vdots\\F_1(F_2(...F_n(x,1)...,1),2(t-\frac{1}{2}))&\text{ if }\frac{1}{2}\leq t\leq 1.
\end{cases}\]
\end{proof}

\begin{exercise}
\begin{enumerate}
\item[(a)] Show that the mapping cylinder of every map $f:S^1\to S^1$ is a CW complex.
\item[(b)] Construct a 2-dimensional CW complex that contains both an annulus $S^1\times I$ and a Mobius band as deformation retracts.
\end{enumerate}
\end{exercise}
\begin{proof}
\begin{enumerate}
\item[(a)]
Observe the following diagram:


\includegraphics[scale=0.5]{Screenshot (1346).png}

We begin with a $1$-skeleton consisting of $e_1^0,e_2^0,e_1^1,e_2^1,e_3^1$. We then introduce a $2$-cell $e_1^2$. Let $\phi:S^1\to X^1$ be the attaching map of $e_1^2$. We define $\phi$ as follows: Identify the arc from $1$ to $2$ with $e_1^0\cup e_1^1$. Then identify the arc from $2$ to $3$ with $e_1^0\cup e_2^1\cup e_2^0$. Then identify the arc from $1$ to $4$ with $e_1^0\cup e_2^1\cup e_2^0$. Finally, let $A$ be the arc from $3$ to $4$. We then have homeomorphisms $g:A/3\sim 4\to S^1$ and $h:S^1\to e_3^1\cup e_2^0$ such that $h\circ f\circ g(3)=e_2^0$, and hence we can attach $A$ to $e_3^1$ under the identification $x\sim h\circ f\circ g(x)\forall x\in A$. This then gives a cell complex that is homeomorphic to $M_f$.
\end{enumerate}
\end{proof}

\begin{exercise}
Show that $S^1*S^1=S^3$, and more generally $S^m*S^n=S^{m+n+1}$.
\end{exercise}
\begin{proof}
We can represent $S^1$ simply as $\{\theta:0\leq\theta<2\pi\}$, and hence $S^1\times S^1\times I\cong \{(\theta_0,\theta_1,r):0\leq\theta_0,\theta_1<2\pi,0\leq r\leq 1\}$. Then, applying the identifications gives\[S^1*S^1\cong\{(\theta_0,\theta_1,r):0\leq\theta_0,\theta_1<2\pi,0\leq r\leq 1\}/\sim,\]where\[(\theta_0,\theta_1,0)\sim(\theta_0,\theta_1',0)\forall \theta_0,\theta_1,\theta_1'\text{ and }(\theta_0,\theta_1,1)\sim(\theta_0',\theta_1,1)\forall \theta_0,\theta_0',\theta_1.\]

Now, $S^3$ is the set of pairs of complex numbers $(z_0,z_1)$ such that the squares of their norms sum to $1$. Representing them in polar coordinates then gives $z_0=r_0 e^{i\theta_0}$ and $z_1=r_1e^{i\theta_1}$, where $r_0^1+r_1^2=1$, $0\leq r_0,r_1\leq 1$ and $0\leq\theta_0,\theta_1<2\pi$. $r_1$ determines $r_0=\sqrt{1-r_1^2}$, so $(z_0,z_1)$ can be associated with a real triple $(\theta_0,\theta_1,r_1)$. However, in order to create a bijection, we need to make the following identifications:\[(\theta_0,\theta_1,0)\sim(\theta_0,\theta_1',0)\forall \theta_0,\theta_1,\theta_1'\] and \[(\theta_0,\theta_1,1)\sim(\theta_0',\theta_1,1)\forall\theta_0,\theta_0',\theta_1.\] These are exactly the identifications which give $S^1*S^1$, and hence $S^1*S^1=S^3$.
\newline

For the general case:

\noindent Recall that $SS^n=S^{n+1}$, and hence note that $S^n*S^0=SS^n=S^{n+1}$, which implies that \[S^n=\overset{n+1\text{ times}}{S^0*...*S^0}.\] Thus \[S^m*S^n=(\overset{n+1\text{ times}}{S^0*...*S^0})*(\overset{m+1\text{ times}}{S^0*...*S^0})=\overset{n+1+m+1\text{ times}}{S^0*...*S^0}=S^{m+n+1}.\]
\end{proof}

\begin{exercise}
Show that the space obtained from $S^2$ by attaching $n$ $2$-cells along any collection of $n$ circles in $S^2$ is homotopy equivalent to the wedge sum of $n+1$ $2$-spheres.
\end{exercise}
\begin{proof}
Let $A$ be the space obtained from $S^2$ by attaching $n$ $2$-cells. We can give $A$ a cell-complex structure as follows: We start with $n$ discs $A_1,...,A_n$, and we then draw lines between them such that the outer circles form part of a perimeter, and every region within the perimeter includes at most one circle $A_i$. Then attach $2$-cells to fill in the area within the perimeter, and another $2$-cell to the whole perimeter. This is shown in the following diagram: 

\includegraphics[scale=0.5]{Screenshot (1351).png}


We can then attach $n$ $2$-cells to the boundaries of the regions $A_1,...,A_n$ to get a space homeomorphic to $A$. Each $A_i$ is a contractible subcomplex, and hence we can collapse each $A_i$ to a point to obtain a space $B$ homotopy-equivalent to $A$, which consists of a sphere $S^2$ with $n$ other spheres attached to it at different points. We can then model $B$ as a cell-complex by starting with the points where the spheres are attached to the first sphere as $0$-cells, then adding $1$-cells to form a connected graph, then attaching $2$-cells to form a sphere, and then attaching $n$ more $2$-cells to each $0$-cell, as shown in the following diagram:

\includegraphics[scale=0.5]{Screenshot (1352).png} 

We can then repeatedly collapse each pair of vertices connected by an edge to a point, so as to end up with $\bigvee_{i=1}^{n+1}S^2$, which will be homotopy equivalent to $A$.
\end{proof}

\begin{exercise}
Show that the subspace $X\subset\mathbb{R}^3$ formed by a Klein bottle intersecting itself in a circle, as shown in the figure, is homotopy equivalent to $S^1\vee S^1\vee S^2$.

\includegraphics[scale=0.5]{Screenshot 2025-08-06 at 01-25-18 AT.dvi - AT.pdf.png}
\end{exercise}
\begin{proof}
First, we can collapse the disc whose boundary is the circle of intersection to a point, to obtain the $2$-sphere with $3$ distinct points identified. This is then homotopy equivalent to the diagram below, since we can collapse both green circles to obtain a sphere with three distinct points identified.

\includegraphics[scale=0.5]{Screenshot (1353).png}

\noindent However, we can also collapse just the lines $x$ and $y$ to points, which then gives $S^1\vee S^1\vee S^2$, as required.
\end{proof}

\end{document}
