\documentclass{article}
\usepackage{graphicx} % Required for inserting images
\usepackage[english]{babel}
\usepackage{amssymb}
\usepackage{amsthm}
\usepackage{enumitem} 
\usepackage{amsmath}
\usepackage{amsfonts}
\usepackage{tikz-cd}
\usetikzlibrary{matrix}
\usepackage[english]{babel}
\usepackage{mathtools}
\usepackage[a4paper, total={6in, 8in}]{geometry}
\usepackage{cite}
\usepackage{graphicx}
\graphicspath{ {./images/} }
\setcounter{section}{-1}

\newtheorem{theorem}{Theorem}[section]
\newtheorem{definition}[theorem]{Definition}
\newtheorem{lemma}[theorem]{Lemma}
\newtheorem{proposition}[theorem]{Proposition}
\newtheorem{corollary}[theorem]{Corollary}
\newtheorem{example}[theorem]{Example}
\newtheorem{remark}[theorem]{Remark}
\newtheorem{exercise}{Exercise}[subsection]


\title{Hatcher Solutions}
\author{Saxon Supple}
\date{July 2025}

\begin{document}
\maketitle
\section{Some Underlying Geometric Notions}
\begin{exercise}
Construct an explicit deformation retraction of the torus with one point deleted onto a graph consisting of two circles intersecting in a point, namely, longitude and meridian circles of the torus.
\end{exercise}
\begin{proof}
We can model the torus as the disk $D^2:=\{(x,y)\in\mathbb{R}^2:x^2+y^2\leq 1\}$ where we split the boundary into $4$ equal arcs and identify opposite arcs under the quotient map $q:D^2\to T$. Let $A:=q(\partial D^2)$ be the subspace we wish to retract $T$ onto. Without loss of generality, suppose that we remove the point $(0,0)$. Given a point $x$, we want to send it to $\frac{x}{\|x\|}$ along the line segment from $x$ to $\frac{x}{\|x\|}$. We thus define the homotopy \[F:T\times I\to T:(q(x),t)\mapsto q\left(x(1-t)+\frac{x}{\|x\|}t\right).\] This homotopes between the identity map and a retraction of $T$ onto $A$ while fixing all points of $A$ throughout, and hence is a deformation retraction.
\end{proof}

\begin{exercise}
Construct an explicit deformation retraction of $\mathbb{R}^n\setminus\{0\}$ onto $S^{n-1}$.
\end{exercise}
\begin{proof}
We define a homotopy \[F:\mathbb{R}^n\setminus\{0\}\times I\to S^{n-1}:(x,t)\mapsto x(1-t)+\frac{x}{\|x\|}t.\] This homotopes between the identity map on $\mathbb{R}^n\setminus\{0\}$ and a retraction of $\mathbb{R}^n\setminus\{0\}$ onto $S^{n-1}$, and hence is a deformation retraction.
\end{proof}

\begin{exercise}
\begin{enumerate}
\item[(a)] Show that the composition of homotopy equivalences $X\to Y$ and $Y\to Z$ is a homotopy equivalence $X\to Z$. Deduce that homotopy equivalence is an equivalence relation.
\item[(b)] Show that the relation of homotopy among maps $X\to Y$ is an equivalence relation.
\item[(c)] Show that a map homotopic to a homotopy equivalence is a homotopy equivalence.
\end{enumerate}
\end{exercise}
\begin{proof}
\begin{enumerate}
\item[(a)] Let $\phi:X\to Y$ and $\psi: Y\to Z$ be homotopy equivalences. There then exist homotopy inverses $\phi_1:Y\to X$ and $\psi_1: Z\to Y$ of $\phi$ and $\psi$ respectively.\[(\phi_1\circ\psi_1)\circ(\psi\circ\phi)= \phi_1\circ(\psi_1\circ\psi)\circ\phi\simeq\phi_1\circ\phi\simeq\text{id}_X\] and\[(\psi\circ\phi)\circ(\phi_1\circ\psi_1)=\psi\circ(\phi\circ\phi_1)\circ\psi_1\simeq\psi\circ\psi_1\simeq\text{id}_Z\]so $\psi\circ\phi$ is a homotopy equivalence.
\item[(b)] Let $\phi,\psi,\rho$ be continuous maps from $X$ to $Y$. Clearly $\phi\simeq\phi$. Furthermore, if $F:X\times I\to Y$ is a homotopy from $\phi$ to $\psi$, then $G:X\times I\to Y:(x,t)\mapsto F(x,1-t)$ is a homotopy from $\psi$ to $\phi$. Finally, let $A:X\times I\to Y$ be a homotopy from $\phi$ to $\psi$ and let $B:X\times I\to Y$ be a homotopy from $\psi$ to $\rho$. We define a map $C:X\times I\to Y$ by \[C(x,t)=\begin{cases}
    A(x,2t)\text{ if }0\leq t\leq \frac{1}{2}\\B(x,2t-1)\text{ if }\frac{1}{2}<t\leq 1.
\end{cases}\]$X\times[0,\frac{1}{2}]$ and $X\times[\frac{1}{2},1]$ are both closed subsets of $X\times I$ with $X\times[0,\frac{1}{2}]\cup X\times[\frac{1}{2},1]=X\times I$. Furthermore, $C_{|{X\times[0,\frac{1}{2}]}}$ and $C_{|{X\times[\frac{1}{2},1]}}$ are both continuous with respect to the induced topologies on $X\times[0,\frac{1}{2}]$ and $X\times[\frac{1}{2},1]$ respectively. Hence, $C$ is continuous on $X\times I$ and is thus a homotopy from $\phi$ to $\rho$.
\item[(c)] Let $\phi:X\to Y$ be a homotopy equivalence with homotopy inverse $\psi:Y\to X$. Let $\phi_1:X\to Y$ be homotopic to $\phi$. Then \[\phi_1\circ\psi\simeq\phi\circ\psi\simeq\text{id}_Y\] and \[\psi\circ\phi_1\simeq\psi\circ\phi\simeq\text{id}_X.\] Hence $\phi_1$ is a homotopy equivalence.
\end{enumerate}
\end{proof}

\begin{exercise}
A $\textbf{deformation retraction in the weak sense}$ of a space $X$ to a subspace $A$ is a homotopy $f_t:X\to X$ such that $f_0=\mathbb{1}$, $f_1(X)\subset A$, and $f_t(A)\subset A$ for all $t$. Show that if $X$ deformation retracts to $A$ in this weak sense, then the inclusion $A\hookrightarrow X$ is a homotopy equivalence.
\end{exercise}
\begin{proof}
Let $\iota:A\to X$ denote the inclusion $A\hookrightarrow X$. Define $\phi:X\to A:x\mapsto f_1(x)$, which is well-defined since $f_1(X)\subset A$. Then \[\phi\circ\iota={f_1}_{|A}\simeq {f_0}_{|A}=\text{id}_A.\]This is a valid homotopy of maps $A\to A$ since $f_1(A)\subset A\forall t$. Furthermore,\[\iota\circ\phi=f_1\simeq\text{id}_X.\] Hence $\iota$ is a homotopy equivalence with homotopy inverse $\phi$.
\end{proof}

\begin{exercise}
Show that if a space $X$ deformation retracts to a point $x\in X$, then for each neighbourhood $U$ of $x$ in $X$ there exists a neighbourhood $V\subset U$ of $x$ such that the inclusion map $V\hookrightarrow U$ is nullhomotopic.
\end{exercise}
\begin{proof}
Without loss of generality, let $U$ be an open neighbourhood of $x$. Define \[F:X\times I\to X:(y,t)\mapsto f_t(y)\] and let $A:=F^{-1}(U)$. $A$ is an open set containing the slice $\{x\}\times I$, so by the Tube Lemma there exists a tube $V\times I\subset X\times I$, where $V$ is open and contains $x$. Furthermore, $V\times I\subset A$, so $V$ consists only of points in $U$ (since $V\times\{0\}\subset U\times\{0\}$) which stay in $U$ throughout the homotopy $f_t$. Let $\iota:V\hookrightarrow U$ be the inclusion map $V\hookrightarrow U$. Then $f_t:X\to X$ restricts to a homotopy ${f_t}_{|V}:V\to U$ and hence $\iota\simeq {f_1}_{|V}$, where ${f_1}_{|V}$ is the constant map, so $\iota$ is nullhomotopic.
\end{proof}

\begin{exercise}
\begin{enumerate}
\item[(a)] Let $X$ be the subspace of $\mathbb{R}^2$ consisting of the horizontal segment $[0,1]\times\{0\}$ together with all the vertical segments $\{r\}\times[0,1-r]$ for $r$ a rational number in $[0,1]$. Show that $X$ deformation retracts to any point in the segment $[0,1]\times\{0\}$, but not to any other point.
\item[(b)] Let $Y$ be the subspace of $\mathbb{R}^2$ that is the union of an infinite number of copies of $X$ arranged as in the figure below. Show that $Y$ is contractible but does not deformation retract onto any point.
\includegraphics[scale=0.5]{Screenshot 2025-07-20 at 19-46-35 AT.dvi - AT.pdf.png}
\item[(c)] Let $Z$ be the zigzag subspace of $Y$ homeomorphic to $\mathbb{R}$ indicated by the heavier line. Show there is a deformation retraction in the weak sense of $Y$ onto $Z$, but no true deformation retraction.
\end{enumerate}
\end{exercise}
\begin{proof}
\begin{enumerate}
\item[(a)] We define a homotopy \[f_t:X\to X:(a,b) \mapsto (a,b(1-t)).\] $f_t$ fixes $[0,1]\times\{0\}$, $f_0=\text{id}_X$ and $f_1$ is a retraction from $X$ to $[0,1]\times\{0\}$, so $f_t$ is a deformation retraction. Hence $[0,1]\times\{0\}$ is a deformation retract of $X$. Any point in $[0,1]\times\{0\}$ is then clearly a deformation retract of $[0,1]\times\{0\}$, so  $X$ deformation retracts to any point in $[0,1]\times\{0\}$ by transitivity. Now suppose $X$ deformation retracts to some other point $x\in X$. We can then find a neighbourhood $U$ of $x$ of the form $B_\epsilon(x)\cap X$ for some $\epsilon > 0$, which does not contain $[0,1]\times\{0\}$. However, $U$ is not path-connected, so there does not exist a neighbourhood $V\subset U$ of $x$ such that the inclusion map $V\hookrightarrow U$ is nullhomotopic; a contradiction.
\item[(b)] Define a homotopy $f_t:Y\to Y$ where $f_t(y)$ sends $y$ a distance of $t$ to the right along $Y$. Then $f_0=\text{id}_Y$ and $f_1(Y)=Z$, where $Z$ is the zigzag subspace of $Y$. $Z$ is contractible, being homeomorphic to $\mathbb{R}$, and hence there exists a homotopy $h_t:Z\to Z$ from $\text{id}_Z$ to a constant function. We can then define a homotopy $F_t:Y\to Y$ of $\text{id}_Y$ to a constant map by\[F_t(y)=\begin{cases}
    f_{2t}(y)\text{ if }0\leq t\leq\frac{1}{2},
    \\h_{2t-1}\circ f_1(y)\text{ if }\frac{1}{2}< t\leq 1.
\end{cases}\]

Now suppose $Y$ deformation retracts onto a point $y$. We can then find a sufficiently small open neighbourhood $U$ of $y$ such that for every neighbourhood $V\subset U$ of $y$, $U$ is not path-connected, and hence the inclusion map $V\hookrightarrow U$ is not nullhomotopic; a contradiction. Hence $Y$ does not deformation retract onto a point $y$.
\item[(c)] Since $Z$ deformation retracts onto a point, if there were a deformation retract of $Y$ onto $Z$, then there would be a deformation retract of $Y$ onto a point. However, (b) showed that this is not the case, and hence $Y$ does not deformation retract onto $Z$. However, the homotopy $f_t$ defined in (b) is a weak deformation retract of $Y$ onto $Z$.
\end{enumerate}
\end{proof}

\begin{exercise}
Fill in the details in the following construction from
[Edwards 1999] of a compact space $Y\subset\mathbb{R}^3$ with the
same properties as the space $Y$ in Exercise $6$, that is, $Y$
is contractible but does not deformation retract to any
point. To begin, let $X$ be the union of an infinite sequence of cones on the Cantor set arranged end-to-end,
as in the figure. Next, form the one-point compactification of $X\times\mathbb{R}$. This embeds in $\mathbb{R}^3$ as a closed disk with curved ‘fins’ attached along circular arcs, and with the one-point compactification of $X$ as a cross-sectional slice.
The desired space $Y$ is then obtained from this subspace of $\mathbb{R}^3$ by wrapping one more
cone on the Cantor set around the boundary of the disk.

\includegraphics[scale=0.5]{Screenshot 2025-07-21 at 17-47-12 AT.dvi - AT.pdf.png}
\end{exercise}

\begin{exercise}
For $n>2$, construct an $n$-room analog of the house with two rooms.
\end{exercise}

\begin{exercise}
Show that a retract of a contractible space is contractible.
\end{exercise}
\begin{proof}
Let $X$ be a contractible space with $A$ a retract of $X$. Let $f_t:X\to X$ be a homotopy from $\text{id}_X$ to a constant map, and let $r:X\to A$ be a retraction of $X$ onto $A$. $r\circ {f_0}_{|A}=\text{id}_A$ and $r\circ {f_1}_{|A}$ is constant due to ${f_1}_{|A}$ being constant. Hence $\text{id}_A$ is homotopic to a constant map so $A$ is contractible.
\end{proof}

\begin{exercise}
Show that a space $X$ is contractible iff every map $f:X\to Y$, for arbitrary $Y$, is
nullhomotopic. Similarly, show $X$ is contractible iff every map $f:Y\to X$ is nullhomotopic.
\end{exercise}
\begin{proof}
$(\impliedby)$: Let $Y=X$ and $f=\text{id}_X$. Then $\text{id}_X$ is nullhomotopic so $X$ is contractible.

$(\implies)$: Let $g_t:X\to X$ be a homotopy from $\text{id}_X$ to a constant map. Then define a homotopy $h_t:X\to Y$ by $h_t(x)=f\circ g_t(x)$. This is a homotopy between $f$ and a constant map, so $f$ is nullhomotopic.

$(\impliedby)$: Again, let $Y=X$ and $f=\text{id}_X$.

$(\implies)$: $g_t\circ f$ is a homotopy from $f$ to a constant map so $f$ is nullhomotopic.
\end{proof}

\begin{exercise}
Show that $f:X\to Y$ is a homotopy equivalence if there exist maps $g,h:Y\to X$ such that $fg\simeq\mathbb{1}$ and $hf\simeq\mathbb{1}$. More generally, show that $f$ is a homotopy equivalence if $fg$ and $hf$ are homotopy equivalences.
\end{exercise}
\begin{proof}
Let $\phi=h\circ f\circ g:Y\to X$. Then\[f\circ\phi=f\circ(h\circ f\circ g)=f\circ(h\circ f)\circ g\simeq f\circ g\simeq\mathbb{1}\] and\[\phi\circ f=(h\circ f\circ g)\circ f=h\circ(f\circ g)\circ f\simeq h\circ f\simeq\mathbb{1}\] so $\phi$ is a homotopy inverse of $f$ so $f$ is a homotopy equivalence.

Let $g_1$ and $h_1$ be the homotopy inverses of $fg$ and $hf$ respectively. We then have that $f(gg_1)\simeq\mathbb{1}$ and $(h_1h)f\simeq\mathbb{1}$, and hence $f$ is a homotopy equivalence with homotopy inverse $h_1hfgg_1$.
\end{proof}

\begin{exercise}
Show that a homotopy equivalence $f:X\to Y$ induces a bijection between the set of path-components of $X$ and the set of path-components of $Y$, and that $f$ restricts to
a homotopy equivalence from each path-component of $X$ to the corresponding path-component of $Y$. Prove also the corresponding statements with components instead
of path-components. Deduce that if the components of a space $X$ coincide with its
path-components, then the same holds for any space $Y$ homotopy equivalent to $X$.
\end{exercise}
\begin{proof}
Let $\sim_X$ be the equivalence relation on $X$ given by $a\sim_X b$ if and only if $a$ and $b$ are in the same path-component. Similarly define the relation $\sim_Y$ on $Y$. Let $g:Y\to X$ be a homotopy inverse of $f$. We let $\phi_t$ and $\psi_t$ be homotopies from $\text{id}_Y$ to $f\circ g$ and from $\text{id}_X$ to $g\circ f$ respectively.


\textbf{We first show that $f$ induces a well-defined map on path components.}\\Let $x_0\sim_X x_1$. Let $\phi:I\to X$ be a path from $x_0$ to $x_1$. Then $f\circ\phi:I\to Y$ is a path from $f(x_0)$ to $f(x_1)$ so $f(x_0)\sim_Y f(x_1)$.

\textbf{We now show that $f$ induces a surjective map on path components.}\\Let $y\in Y$. Then $y=\phi_0(y)\sim_Y \phi_1(y)=f(g(y))$, and hence $y$ is in the same path component as a point in the image of $f$.

\textbf{We finally show that $f$ induces an injective map on path components.}\\ Suppose that $f(x_0)\sim_Yf(x_1)$. Then $g\circ f(x_0)\sim_X g\circ f(x_1)$ and hence $\psi_1(x_0)\sim_X\psi_1(x_1)$. We also have that $x_0=\psi_0(x_0)\sim_X\psi_1(x_0)$ and $x_1=\psi_0(x_1)\sim_X\psi_1(x_1)$, and so by transitivity\[x_0\sim_X\psi_1(x_0)\sim_X\psi_1(x_1)\sim_X x_1\implies x_0\sim_X x_1.\]Hence $f$ induces a bijection on path-components.\newline

Let $[x]_X$ be the path component of some point $x\in X$ and let $[f(x)]_Y$ be the path component in $Y$ of $f(x)$. $f$ then restricts to a map \[f_{|[x]_X}:[x]_X\to [f(x)]_Y.\] Furthermore, $g\circ f(x)\sim_X x$, and hence $g$ also restricts to a map \[g_{|[f(x)]_Y}:[f(x)]_Y\to[x]_X.\] We can then also restrict the homotopies $\phi_t$ and $\psi_t$ to \[\phi_{t|[f(x)]_Y}:[f(x)]_Y\to[f(x)]_Y\] and \[\psi_{t|[x]_X}:[x]_X\to[x]_X\] respectively, since $\forall x_0\in[x]_X$ we have $\psi_t(x_0)\sim_Xx_0\forall t$, and $\forall y_0\in[f(x)]_Y$ we have $\phi_t(y_0)\sim_Y y_0\forall t$. Hence $f_{|[x]_X}$ is a homotopy equivalence between $[x]_X$ and $[f(x)]_Y$.

Now let $\sim_X$ and $\sim_Y$ be the corresponding relations for connected components.

\textbf{We first show that $f$ induces a well-defined map on connected components.}
Let $x_0\sim_X x_1$. Let $Z$ be a connected subspace containing $x_0$ and $x_1$. The image of a connected space is connected, and hence $f(x_0)\sim_Y f(x_1)$, as both are contained in $f(Z)$.

\textbf{We next show that $f$ induces a surjective map on connected components.}
Let $y\in Y$. Let $[y]_Y$ be the connected component of $y$. As show earlier, $y$ is in the same path component as $f\circ g(y)$, and hence is in the same connected component as $f\circ g(y)$. Hence, $[y]_Y$ is in the image of the map on connected components induced by $f$.

\textbf{We finally show that $f$ induces an injective map on connected components.}
Let $f(x_0)\sim_Yf(x_1)$. Then $g\circ f(x_0)\sim_X g\circ f(x_1)$. We then have that $x_0$ is in the same path-component as $g\circ f(x_0)$ and $x_1$ is in the same path-component as $g\circ f(x_1)$, and hence that $x_0$ is in the same connected component as $g\circ f(x_0)$ and $x_1$ is in the same connected component as $g\circ f(x_1)$. Hence by transitivity\[x_0\sim_X g\circ f(x_0)\sim_Xg\circ f(x_1)\sim_X x_1\implies x_0\sim_X x_1.\] Hence $f$ induces a bijection on connected components.

Let $[x]_X$ be the connected component of some point $x\in X$ and let $[f(x)]_Y$ be the connected component of $f(x)$ in $Y$. $f$ then restricts to a map\[f_{|[x]_X}:[x]_X\to [f(x)]_Y.\]Furthermore, $g\circ f(x)\sim_X x$, and hence $g$ also restricts to a map \[g_{|[f(x)]_Y}:[f(x)]_Y\to[x]_X.\]We can then also restrict the homotopies $\phi_t$ and $\psi_t$ to \[\phi_{t|[f(x)]_Y}:[f(x)]_Y\to[f(x)]_Y\] and \[\psi_{t|[x]_X}:[x]_X\to[x]_X\] respectively, since $\forall x_0\in[x]_X$ we have $\psi_t(x_0)\sim_Xx_0\forall t$, and $\forall y_0\in[f(x)]_Y$ we have $\phi_t(y_0)\sim_Y y_0\forall t$. Hence $f_{|[x]_X}$ is a homotopy equivalence between $[x]_X$ and $[f(x)]_Y$.

Finally, suppose that the connected components of a space $X$ coincide with its path-components, and let $Y$ be homotopy equivalent to $X$. Let $f:X\to Y$ be a homotopy equivalence. Let \[f_\#:\{\text{Path/connected components of } X\}\to\{\text{Path components of } Y\}\] and \[f_*:\{\text{Path/connected components of } X\}\to\{\text{Connected components of } Y\}\] be the maps induced by $f$. We then have a bijection between path-components of $Y$ and connected components of $Y$ given by $f_*\circ f_\#^{-1}$. Furthermore, given any path-component $A\subseteq Y$, $f_*\circ f_\#^{-1}(A)$ is the connected component containing $A$. Hence there must be at most one path-component per connected component, since otherwise multiple path-components would be mapped to the same connected component by $f_*\circ f_\#^{-1}$, and it would cease to be injective. Hence the connected components of $Y$ coincide with its path-components.
\end{proof}

\begin{exercise}
Show that any two deformation retractions $r_t^0$ and $r_t^1$ of a space $X$ onto a subspace $A$ can be joined by a continuous family of deformation retractions $r_t^s$, $0\leq s\leq 1$, of $X$ onto $A$, where continuity means that the map $X\times I\times I\to X$ sending $(x,s,t)$ to $r_t^s(x)$ is continuous.
\end{exercise}
\begin{proof}
We know what $r_t^s$ is on $X\times\{0,1\}\times I\cup A\times I\times I\cup X\times I\times\{0\}$.

\noindent Define $r_t^s$ by\[r_t^s(x)=\begin{cases}
    r_t^0\circ r_{2ts}^1(x)\text{ if } 0\leq s <\frac{1}{2},\\r_{2t(1-s)}^0\circ r_t^1(x)\text{ if }\frac{1}{2}\leq s\leq 1.
\end{cases}\] If $s=0$, then $r_t^0=r_t^0\circ r_0^1=r_t^0$.

\noindent If $s=1$, then $r_t^1=r_0^0\circ r_t^1=r_t^1$.

\noindent If $a\in A$, then $r_t^s(a)=a\forall s,t$.

\noindent If $t=0$, then $r_0^s=\text{id}_X\forall s$.

\noindent If $t=1$, then $r_1^0(x)\in A\forall x$ and $r_1^1(x)\in A\forall x$, and hence $r_1^s(x)\in A\forall x$.
\end{proof}

\begin{exercise}
Given positive integers $v$, $e$ and $f$ satisfying $v-e+f=2$, construct a cell structure on $S^2$ having $v$ $0$-cells, $e$ $1$-cells, and $f$ $2$-cells.
\end{exercise}
\begin{proof}
For $(v,e,f)=(2+m,m+n+1,n+1)$ with $m>0,n>1$, we can use the following construction: Begin with $0$-cells $e_1^0,...,e_{2+m}^0$. Then attach $n-2$ loops $e_1^1,...,e_{n-2}^1$ to $e_1^0$, and add two $1$-cells $e_{n-1}^1$ and $e_n^1$, each with one endpoint attached to $e_1^0$ and the other endpoint attached to $e_{2+m}^0$. Then add $1$-cells $e_{n+1}^1,...,e_{n+1+m}^1$, where $e_{n+i}^1$ is attached to $e_i^0$ and $e_{i+1}^0$. Then add $2$-cells $e_1^2,...,e_{n+1}^2$, where the first $n-2$ are attached to the $n-2$ loops, the next two are attached to $e_1^0\cup...\cup e_{2+m}^0\cup e_{n-1}^1\cup e_{n+1}^1\cup...e_{n+1+m}^1$ and $e_1^0\cup...\cup e_{2+m}^0\cup e_n^1\cup e_{n+1}^1\cup...e_{n+1+m}^1$ respectively. Finally, attach the last $2$-cell to $e_1^0\cup e_2^0\cup e_1^1\cup...\cup e_n^1$.

\includegraphics[scale=0.5]{Screenshot (1340).png}

If $v=2$, then $e=f$. First suppose $e>1$. We can then form $S^2$ by starting with $e_1^0$ and $e_2^0$, and then attaching $e-2$ loops to $e_1^0$ and two $1$-cells each connecting the two $0$-cells, and finally attaching $e$ $2$-cells.

\includegraphics[scale=0.5]{Screenshot (1341).png}

The case $(v,e,f)=(2,1,1)$ is then just the standard construction of starting with two points, attaching two lines joining them to form a circle, filling in the centre of the circle, and then identifying the two lines.

Now consider the case $(v,e,f)=(2+m,m+2,2)$. That is, the case where $n=1$. The following diagram works:

\includegraphics[scale=0.5]{Screenshot (1342).png}

For the case $(v,e,f)=(2+m,m+1,1)$, which corresponds to $n=0$, the following diagram works:

\includegraphics[scale=0.5]{Screenshot (1343).png}

For the case $(v,e,f)=(1,x,x+1)$, the following diagram works:

\includegraphics[scale=0.5]{Screenshot (1344).png}
\end{proof}

\begin{exercise}
Enumerate all the subcomplexes of $S^\infty$, with the cell structure on $S^\infty$ that has $S^n$ as its $n$-skeleton.
\end{exercise}
\begin{proof}
$X^n=S^n$ is a subcomplex of $S^\infty$ for every $n$. Also, for each $X^n=S^n$, $X^n\cup e_{1}^{n+1}$ and $X^n\cup e_{2}^{n+1}$ are subcomplexes. It is not possible to have a subcomplex formed by a proper subset of $X^n$ and some $\emptyset\neq\phi\subseteq X^{n+i}\setminus X^n$ for some $i\geq 1$, since $\phi$ attaches to the whole of $X^n$, and hence the subcomplex, being closed, would contain $X^n$; a contradiction. Hence, if a subcomplex contains an $i$-cell for $i>0$, it must also contain $X^{i-1}$. Hence the only possible subcomplexes are:
\[\emptyset, S^\infty, e_1^0,e_2^0, S^{i-1}\cup e_1^i,S^{i-1}\cup e_2^i \text{ for } i>0,\text{ and } S^n\text{ for }n\geq 0.\]
\end{proof}

\begin{exercise}
Show that $S^\infty$ is contractible.
\end{exercise}
\begin{proof}
For $k>0$, each $k$-skeleton $X^k\subseteq S^\infty$ is $S^{k-1}\cup e_1^k\cup e_2^k$, where $S^{k-1}\cup e_1^k$ and $S^{k-1}\cup e_2^k$ are both the disk $D^k$. Furthermore, for $k>1$, each $S^{k-1}\cup e_i^k$ can be deformation retracted into either $X^{k-2}\cup e_1^{k-1}$ or $X^{k-2}\cup e_2^{k-1}$. We can then repeat this recursively until we reach $X^0\cup e_i^{1}$, which we can then deformation retract to a single point in $X^0$. For $k>1$, let \[F_k:S^{k-1}\cup e_1^k\times I\to S^{k-1}\cup e_1^k\] be a deformation retraction of $S^{k-1}\cup e_1^k$ onto $X^{k-2}\cup e_1^{k-1}$, and for $k=1$, let \[F_1:X^0\cup e_1^1\times I\to X^0\cup e_1^1\] be a deformation retraction of $X^0\cup e_1^1$ onto $e_1^0$. Then, given an $x\in S^\infty$, let $n$ be the smallest positive integer such that $x\in X^{n-1}\cup e_1^n$. We then define a homotopy $h_t$ between $\text{id}_{S^\infty}$ and a constant map by\[h_t(x)=\begin{cases}
    x&\text{if }0\leq t<\frac{1}{2^{n+1}},\\F_n(x,2^n(t-\frac{1}{2^{n+1}}))&\text{if }\frac{1}{2^{n+1}}\leq t<\frac{1}{2^n},\\F_{n-1}(F_n(x,1),2^{n-1}(t-\frac{1}{2^{n}}))&\text{if }\frac{1}{2^n}\leq t<\frac{1}{2^{n-1}},\\\vdots&\vdots\\F_1(F_2(...F_n(x,1)...,1),2(t-\frac{1}{2}))&\text{if }\frac{1}{2}\leq t\leq 1.
\end{cases}\]Since CW complexes have the weak topology with respect to their skeleta, a map is continuous if and only if its restriction to each skeleton is continuous. This is clearly the case for $h_t$, and hence $h_t$ is continuous.
\end{proof}

\begin{exercise}
\begin{enumerate}
\item[(a)] Show that the mapping cylinder of every map $f:S^1\to S^1$ is a CW complex.
\item[(b)] Construct a 2-dimensional CW complex that contains both an annulus $S^1\times I$ and a M\"obius band as deformation retracts.
\end{enumerate}
\end{exercise}
\begin{proof}
\begin{enumerate}
\item[(a)]
Observe the following diagram:


\includegraphics[scale=0.5]{Screenshot (1346).png}

We begin with a $1$-skeleton consisting of $e_1^0,e_2^0,e_1^1,e_2^1,e_3^1$. We then introduce a $2$-cell $e_1^2$. Let $\phi:S^1\to X^1$ be the attaching map of $e_1^2$. We define $\phi$ as follows: Identify the arc from $1$ to $2$ with $e_1^0\cup e_1^1$. Then identify the arc from $2$ to $3$ with $e_1^0\cup e_2^1\cup e_2^0$. Then identify the arc from $1$ to $4$ with $e_1^0\cup e_2^1\cup e_2^0$. Finally, let $A$ be the arc from $3$ to $4$. We then have homeomorphisms $g:A/3\sim 4\to S^1$ and $h:S^1\to e_3^1\cup e_2^0$ such that $h\circ f\circ g(3)=e_2^0$, and hence we can attach $A$ to $e_3^1$ under the identification $x\sim h\circ f\circ g(x)\forall x\in A$. This then gives a cell complex that is homeomorphic to $M_f$.
\item[(b)] We can form a CW complex out of gluing $S^1\times I$ to the midcircle of the M\"obius band, as in the diagram below:


\includegraphics[scale=0.5]{Screenshot (1366).png}\newline We can then deformation retract $S^1\times I$ onto the midcircle of the M\"obius band, and we can deformation retract the M\"obius band onto $S^1\times I$ by deformation retracting it onto its midcircle, which is identified with $S^1\times\{0\}$.
\end{enumerate}
\end{proof}

\begin{exercise}
Show that $S^1*S^1=S^3$, and more generally $S^m*S^n=S^{m+n+1}$.
\end{exercise}
\begin{proof}
We can represent $S^1$ simply as $\{\theta:0\leq\theta<2\pi\}$, and hence $S^1\times S^1\times I\cong \{(\theta_0,\theta_1,r):0\leq\theta_0,\theta_1<2\pi,0\leq r\leq 1\}$. Then, applying the identifications gives\[S^1*S^1\cong\{(\theta_0,\theta_1,r):0\leq\theta_0,\theta_1<2\pi,0\leq r\leq 1\}/\sim,\]where\[(\theta_0,\theta_1,0)\sim(\theta_0,\theta_1',0)\forall \theta_0,\theta_1,\theta_1'\text{ and }(\theta_0,\theta_1,1)\sim(\theta_0',\theta_1,1)\forall \theta_0,\theta_0',\theta_1.\]

Now, $S^3$ is the set of pairs of complex numbers $(z_0,z_1)$ such that the squares of their norms sum to $1$. Representing them in polar coordinates then gives $z_0=r_0 e^{i\theta_0}$ and $z_1=r_1e^{i\theta_1}$, where $r_0^1+r_1^2=1$, $0\leq r_0,r_1\leq 1$ and $0\leq\theta_0,\theta_1<2\pi$. $r_1$ determines $r_0=\sqrt{1-r_1^2}$, so $(z_0,z_1)$ can be associated with a real triple $(\theta_0,\theta_1,r_1)$. However, in order to create a bijection, we need to make the following identifications:\[(\theta_0,\theta_1,0)\sim(\theta_0,\theta_1',0)\forall \theta_0,\theta_1,\theta_1'\] and \[(\theta_0,\theta_1,1)\sim(\theta_0',\theta_1,1)\forall\theta_0,\theta_0',\theta_1.\] These are exactly the identifications which give $S^1*S^1$, and hence $S^1*S^1=S^3$.
\newline

For the general case:

\noindent Recall that $SS^n=S^{n+1}$, and hence note that $S^n*S^0=SS^n=S^{n+1}$, which implies that \[S^n=\overset{n+1\text{ times}}{S^0*...*S^0}.\] Thus \[S^m*S^n=(\overset{n+1\text{ times}}{S^0*...*S^0})*(\overset{m+1\text{ times}}{S^0*...*S^0})=\overset{n+1+m+1\text{ times}}{S^0*...*S^0}=S^{m+n+1}.\]
\end{proof}

\begin{exercise}
Show that the space obtained from $S^2$ by attaching $n$ $2$-cells along any collection of $n$ circles in $S^2$ is homotopy equivalent to the wedge sum of $n+1$ $2$-spheres.
\end{exercise}
\begin{proof}
Let $A$ be the space obtained from $S^2$ by attaching $n$ $2$-cells. We can give $A$ a cell-complex structure as follows: We start with $n$ discs $A_1,...,A_n$, and we then draw lines between them such that the outer circles form part of a perimeter, and every region within the perimeter includes at most one circle $A_i$. Then attach $2$-cells to fill in the area within the perimeter, and another $2$-cell to the whole perimeter. This is shown in the following diagram: 

\includegraphics[scale=0.5]{Screenshot (1351).png}


We can then attach $n$ $2$-cells to the boundaries of the regions $A_1,...,A_n$ to get a space homeomorphic to $A$. Each $A_i$ is a contractible subcomplex, and hence we can collapse each $A_i$ to a point to obtain a space $B$ homotopy-equivalent to $A$, which consists of a sphere $S^2$ with $n$ other spheres attached to it at different points. We can then model $B$ as a cell-complex by starting with the points where the spheres are attached to the first sphere as $0$-cells, then adding $1$-cells to form a connected graph, then attaching $2$-cells to form a sphere, and then attaching $n$ more $2$-cells to each $0$-cell, as shown in the following diagram:

\includegraphics[scale=0.5]{Screenshot (1352).png} 

We can then repeatedly collapse each pair of vertices connected by an edge to a point, so as to end up with $\bigvee_{i=1}^{n+1}S^2$, which will be homotopy equivalent to $A$.
\end{proof}

\begin{exercise}
Show that the subspace $X\subset\mathbb{R}^3$ formed by a Klein bottle intersecting itself in a circle, as shown in the figure, is homotopy equivalent to $S^1\vee S^1\vee S^2$.

\includegraphics[scale=0.5]{Screenshot 2025-08-06 at 01-25-18 AT.dvi - AT.pdf.png}
\end{exercise}
\begin{proof}
First, we can collapse the disc whose boundary is the circle of intersection to a point, to obtain the $2$-sphere with $3$ distinct points identified. This is then homotopy equivalent to the diagram below, since we can collapse both segments $a$ and $b$ to obtain a sphere with three distinct points identified.

\includegraphics[scale=0.5]{Screenshot (1355).png}

\noindent However, we can also collapse the segments $x$ and $y$ instead, which then gives $S^1\vee S^1\vee S^2$, as required.
\end{proof}

\begin{exercise}
If $X$ is a connected Hausdorff space that is a union of a finite number of $2$-spheres, any two of which intersect in at most one point, show that $X$ is homotopy equivalent to a wedge sum of $S^1$'s and $S^2$'s.
\end{exercise}
\begin{proof}
Firstly, we can represent the spheres as a simple connected graph, where each vertex represents a sphere, and two vertices being connected by an edge represents two spheres intersecting at a point. Let $T$ be a spanning tree of the graph. We can construct a CW complex of $X$ as follows: For each sphere $X_i$, let $A_i$ be the set of points of intersection of $X_i$ with other spheres which are connected to $X_i$ by an edge in $T$, and let $B_i$ be the set of points of intersection of $X_i$ with other spheres which are connected to $X_i$ by an edge not in $T$. Let $C_i$ be a copy of $B_i$, so that $\left|\bigcup_{i=1}^nC_i\right|=|\bigsqcup_{i=1}^nB_i|$. We then form a $0$-skeleton consisting of $0$-cells which correspond to the points in $A_1\cup...\cup A_n$ and $C_1\cup...\cup C_n$. For each $i$, we connect all the points in $A_i\cup C_i$ by $1$-cells, so as to form a simple acyclic graph, which we shall denote as $G_i$. We then connect each pair of points in $C_i\cup C_j$ which correspond to a single point in $B_i$ with a $1$-cell. Finally, we attach a $2$-cell to each $G_i$, which then gives a space that is homotopy equivalent to $X$; for we can collapse the $1$-cells connecting the points in $C_1\cup...\cup C_n$ to obtain all the intersections. This is illustrated in the following diagram, where the black circles represent $2$-spheres, the green dots represent points in $C_1\cup...\cup C_n$, the red dots represent points in $A_1\cup...\cup A_n$, the purple lines represent $G_i$'s, and the blue line represents a $1$ cell connecting to points in $C_1\cup...\cup C_n$.

\includegraphics[scale=0.5]{Screenshot (1361).png}

If we instead collapse each $G_i$ to a point, we then obtain a wedge sum of $n$ $2$-spheres, along with the loops corresponding to the $1$-cells connecting the points in $C_1\cup...\cup C_n$, which are now all the same point.
\end{proof}

\begin{exercise}
Let $X$ be a finite graph lying in a half-plane $P\subseteq \mathbb{R}^3$ and intersecting the edge of $P$ in a subset of the vertices of $X$. Describe the homotopy type of the 'surface of revolution' obtained by rotating $X$ about the edge line of $P$.
\end{exercise}
\begin{proof}
First, assume that $X$ is connected. Without loss of generality, we can assume that none of the edges of $X$ intersect, for if they do, then we can simply add a vertex at the point of intersection. We can also assume without loss of generality that there are no edges connecting two vertices on $\partial P$, for if there is one, then we can add a vertex in the middle of the edge. We first consider three subgraphs of $X$:
\begin{enumerate}
\item The first, which we shall denote as $T$, is a spanning forest of the collection of vertices of $X$ which do not lie in $\partial P$.
\item The second, which we shall denote as $A$, consists of all the edges of $X$ which are not part of $T$, and which do not have an endpoint on $\partial P$, along with their endpoints.
\item The third, which we shall denote as $B$, consists of all vertices on $\partial P$, all vertices of $X$ which are connected to a vertex on $\partial P$ by an edge, and all edges which have an endpoint in $\partial P$.
\end{enumerate} Let $T_1,...,T_n$ be the spanning trees which comprise $T$. Let $V_{T_i}$ be the subcomplex of $T_i$ formed by all the vertices of $T_i$, let $v_i$ be a vertex in $T_i$ that is connected to a vertex in $\partial P$ by an edge, let $V_A$ be all the vertices of $A$, and let $V_B:=\{w_1,...,w_m\}$ be all vertices in $B$ which do not lie on $\partial P$. If we define a map $f:V_A\sqcup V_B\to T$ which maps the vertices of $V_A\sqcup V_B$ to their appropriate vertices in $T$, we then have that $X\cong T\sqcup_f(A\sqcup B)$. We can then define a map $g:V_A\sqcup V_B\to T$ which maps every vertex of $V_A\sqcup V_B$ to some $v_i$. Clearly $f\simeq g$. Let \[F:(V_A\sqcup V_B)\times I\to T\] be a homotopy from $f$ to $g$. We can then obtain a homotopy \[G:(V_A\sqcup V_B)\times S^1\times I\to T\times S^1:(v,\omega,t)\mapsto(F(v,t),\omega).\] Continuity of $G$ follows from the universal property of the product topology, for \[\pi_T\circ G:(V_A\sqcup V_B)\times S^1\times I\to T:(v,\omega,t)\mapsto F(v,t)\] and \[\pi_{S^1}\circ G:(V_A\sqcup V_B)\times S^1\times I\to S^1:(v,\omega,t)\mapsto \omega\] are both continuous. Define \[\alpha:(V_A\sqcup V_B)\times S^1\to T\times S^1:(v,\omega)\mapsto G(v,\omega,0)\] and \[\beta:(V_A\sqcup V_B)\times S^1\to T\times S^1:(v,\omega)\mapsto G(v,\omega,1).\] Let $Y$ be the resulting space after rotating $X$ about $\partial P$, which is homeomorphic to \[(X\times S^1)/\sim,\] where $\sim$ is given by $(v,\omega)\sim(v,\omega')\forall \omega, v\in X^0\cap \partial P$. Then if we define a relation $\sim_B$ on $(A\sqcup B)\times S^1$ given by $(v,\omega)\sim_B(v,\omega')\forall v\in \partial P,\omega\in S^1$, we have \begin{align*}Y&\cong (T\times S^1)\sqcup_\alpha(((A\sqcup B)\times S^1)/\sim_B)\\&\simeq(T\times S^1)\sqcup_\beta(((A\sqcup B)\times S^1)/\sim_B)\\&\simeq(\{v_1,...,v_n\}\times S^1)\sqcup_{r\circ\beta}(((A\sqcup B)\times S^1)/\sim_B),\end{align*} where the last homotopy equivalence was obtained by deformation retracting $T\times S^1$ to $\{v_1,...,v_n\}\times S^1$, and letting $r:T\times S^1\to \{v_1,...,v_n\}\times S^1$ be a retraction. Let $q$ be the number of edges in $A$. Then the joining of $A$ to $\{v_1,...,v_n\}\times S^1$ amounts to attaching $q$ $2$-tori onto $\{v_1,...,v_n\}\times S^1$ around a longitude. Finally, for each $i$, we choose a subspace $B_i$ which consists of $\{v_i\}\times S^1$, a vertex in $\partial P$ which is connected to $v_i$ by a disk, and the connecting disk itself. Each $B_i$ is a contractible subcomplex, and hence we can collapse it to a point. Then all the disks formed from the edges of $B$ will become $2$-spheres or horn tori, and the tori formed from the edges of $A$ will also become horn tori. Moreover, each horn torus, being formed by a loop with a basepoint on $\partial P$ being rotated about $\partial P$, is homeomorphic to semicircle (with the line of the semicircle lying on $\partial P$) being rotated around $\partial P$ and then having the line collapsed to a point. That is in turn homotopy equivalent to $S^2\vee S^1$, for we can move one of the attaching points of the line to the other point. This is illustrated in the diagram below:

\includegraphics[scale=0.5]{Screenshot (1377).png}

\noindent We finish by repeating the following procedure: We choose a pair of vertices on $\partial P$ which are the endpoints of a sphere, and we choose one such sphere and collapse a contractible arc on the sphere connecting those two vertices. This will maintain the homotopy type of the sphere, while identifying two points on the other such spheres, which then gives them the homotopy types of $S^2\vee S^1$. We repeat this until every vertex is identified with every other, which is possible, due to $X$ being connected. The end result is a wedge sum of circles and spheres.
\end{proof}

\begin{exercise}
Show that a CW complex is contractible if it is the union of two contractible subcomplexes whose intersection is also contractible.
\end{exercise}
\begin{proof}
Let $X$ be a CW complex which is a union of contractible subcomplexes $A$ and $B$, where $A\cap B$ is also contractible. Then $X\simeq X/A\cong B/(A\cap B)\simeq B\simeq\{\text{pt}\}$.

\noindent We can observe the homeomorphism $X/A\cong B/(A\cap B)$ as follows:

Let $a\in A\cap B$. Let $q_X:X\to X/A$ and $q_B:B\to B/(A\cap B)$ be the quotient maps, and define $f:X/A\to B/(A\cap B)$ by\[f([x]_X)=\begin{cases}
    [a]_B&\text{if }x\in A,\\
    [x]_B&\text{if }x\in B.
\end{cases}\] Next, define $g:X\to B/(A\cap B)$ by\[g(x)=\begin{cases}
    [a]_B&\text{if }x\in A,\\
    [x]_B&\text{if }x\in B.
\end{cases}\]These maps are both well-defined, for if $x\in A\cap B$, then $[x]_B=[a]_B$. Furthermore, g is continuous by the pasting lemma, and $g=f\circ q_X$, so $f$ is continuous by the universal property of the final topology.

Now define $f^{-1}:B/(A\cap B)\to X/A$ by\[f^{-1}([x]_B)=\begin{cases}
    [a]_X&\text{if }x\in A,\\
    [x]_X&\text{if }x\in B.
\end{cases}\]and define $h:B\to X/A$ by\[h(x)=\begin{cases}
    [a]_X&\text{if }x\in A,\\
    [x]_X&\text{if }x\in B.
\end{cases}\]Again, these maps are well-defined, for if $x\in A\cap B$, then $[x]_X=[a]_X$. Furthermore, $h$ is continuous, due to simply being the composition of $q_X$ with the inclusion map $B\hookrightarrow X$. Hence, $f^{-1}$ is continuous by the universal property of the final topology, for $h=f^{-1}\circ q_B$. $f$ and $f^{-1}$ are two-sided inverses, and hence $f$ is a homeomorphism.
\end{proof}

\begin{exercise}
Let $X$ and $Y$ be CW complexes with $0$-cells $x_0$ and $y_0$. Show that the quotient spaces $X*Y/(X*\{y_0\}\cup\{x_0\}*Y)$ and $S(X\wedge Y)/S(\{x_0\}\wedge\{y_0\})$ are homeomorphic, and deduce that $X*Y\simeq S(X\wedge Y)$.
\end{exercise}
\begin{proof}
\begin{align*}\{x_0\}\wedge\{y_0\}&=(\{x_0\}\times\{y_0\})/(\{x_0\}\times\{y_0\}\cup\{x_0\}\times\{y_0\})\\&=\{(x_0,y_0)\}/\{(x_0,y_0)\}\\&=\{(x_0,y_0)\}.\end{align*}and hence\[S(\{x_0\}\wedge\{y_0\})=S\{(x_0,y_0)\}=\{(x_0,y_0)\}\times I.\]
Furthermore,
\begin{align*}
S(X\wedge Y)&=S(X\times Y/(\{x_0\}\times Y\cup X\times\{y_0\}))\\&=X\times Y\times I/\sim
\end{align*}where \begin{align*}(x,y,0)&\sim(x',y',0)&\forall x,y,x',y',\\ (x,y,1)&\sim(x',y',1)&\forall x,y,x',y',\\(x_0,y,t)&\sim(x_0,y_0,t)&\forall y,t,\\ (x,y_0,t)&\sim(x_0,y_0,t)&\forall x,t.\end{align*} Hence \[S(X\wedge Y)/S(\{x_0\}\wedge\{y_0\})=X\times Y\times I/\sim\]where\begin{align*}(x,y,0)&\sim(x',y',0)&\forall x,y,x',y',\\ (x,y,1)&\sim(x',y',1)&\forall x,y,x',y',\\(x_0,y,t)&\sim(x_0,y_0,0)&\forall y,t,\\ (x,y_0,t)&\sim(x_0,y_0,0)&\forall x,t.\end{align*} which simplifies to \begin{align*}(x,y,0)&\sim(x_0,y_0,0)&\forall x,y,\\ (x,y,1)&\sim(x_0,y_0,0)&\forall x,y,\\(x_0,y,t)&\sim(x_0,y_0,0)&\forall y,t,\\ (x,y_0,t)&\sim(x_0,y_0,0)&\forall x,t.\end{align*}Next,
\[X*Y=X\times Y\times I/\sim\] where \begin{align*}
(x,y,0)&\sim(x,y',0)&\forall x,y,y'\\(x,y,1)&\sim(x',y,1)&\forall x,x',y
\end{align*}Furthermore, \[X*\{y_0\}\cup \{x_0\}*Y=(X\times\{y_0\}\times I\cup \{x_0\}\times Y\times I)/\sim\]where\begin{align*}
(x_0,y,0)&\sim(x_0,y',0)&\forall y,y'\\(x,y_0,1)&\sim(x',y_0,1)&\forall x,x'
\end{align*}and hence\[X*Y/(X*\{y_0\}\cup\{x_0\}*Y)=X\times Y\times I/\sim\]where\begin{align*}
(x,y,0)&\sim(x,y',0)&\forall x,y,y'\\(x,y,1)&\sim(x',y,1)&\forall x,x',y\\(x,y_0,t)&\sim(x_0,y_0,0)&\forall x_,t\\(x_0,y,t)&\sim(x_0,y_0,0)&\forall y,t\\(x_0,y,0)&\sim(x_0,y',0)&\forall y,y'\\(x,y_0,1)&\sim(x',y_0,1)&\forall x,x'
\end{align*} which simplifies to \begin{align*}
(x,y,0)&\sim(x,y',0)&\forall x,y,y'\\(x,y,1)&\sim(x',y,1)&\forall x,x',y\\(x,y_0,t)&\sim(x_0,y_0,0)&\forall x_,t\\(x_0,y,t)&\sim(x_0,y_0,0)&\forall y,t
\end{align*}Furthermore,\[(x,y,0)\sim(x,y_0,0)\sim(x_0,y_0,0)\forall x,y,\] and \[(x,y,1)\sim(x_0,y,1)\sim(x_0,y_0,0)\forall x,y.\] So the final simplification is\begin{align*}
(x,y,0)&\sim(x_0,y_0,0)&\forall x,y\\(x,y,1)&\sim(x_0,y_0,1)&\forall x,y\\(x,y_0,t)&\sim(x_0,y_0,0)&\forall x_,t\\(x_0,y,t)&\sim(x_0,y_0,0)&\forall y,t
\end{align*}These identifications are identical to the identifications which yield $S(X\wedge Y)/S(\{x_0\}\wedge\{y_0\})$, and hence \[X*Y/(X*\{y_0\}\cup\{x_0\}*Y)\cong S(X\wedge Y)/S(\{x_0\}\wedge\{y_0\}).\]
Finally, both $(X*\{y_0\}\cup\{x_0\}*Y)$ and $S(\{x_0\}\wedge\{y_0\})$ are contractible subcomplexes, and hence\[X*Y\simeq X*Y/(X*\{y_0\}\cup\{x_0\}*Y)\cong S(X\wedge Y)/S(\{x_0\}\wedge\{y_0\})\simeq S(X\wedge Y).\]
\end{proof}

\begin{exercise}
If $X$ is a CW complex with components $X_\alpha$, show that the suspension $SX$ is homotopy equivalent to $Y\bigvee_\alpha SX_\alpha$ for some graph $Y$. In the case that $X$ is a finite graph, show that $SX$ is homotopy equivalent to a wedge sum of circles and $2$-spheres.
\end{exercise}
\begin{proof}
$SX\cong(\coprod_\alpha X_\alpha\times I)/\sim$, where\begin{align*}
\forall\alpha:(x,0)&\sim(x',0)&\forall x,x'\in X_\alpha\\\forall\alpha:(x,1)&\sim(x',1)&\forall x,x'\in X_\alpha
\end{align*} and hence if we select an $x_\alpha\in X_\alpha\forall\alpha$, we have $SX\cong(\coprod_\alpha SX_\alpha)/\sim$ where\begin{align*}
[(x_\alpha,0)]&\sim[(x_{\alpha'},0)]&\forall \alpha,\alpha'\\ [(x_\alpha,1)]&\sim[(x_{\alpha'},1)]&\forall \alpha,\alpha'
\end{align*} This can then be expressed as the quotient of a wedge sum with chosen points $[(x_\alpha,0)]$ as follows:\[\left.SX\cong\left(\bigvee_\alpha SX_\alpha\right)\middle/\sim\right.\] where\[[(x_\alpha,1)]\sim[(x_{\alpha'},1)]\forall \alpha,\alpha'.\]Now let $G$ be a connected acyclic graph with vertices $v_\alpha$. Call the set of vertices the subcomplex $A$. Then define $f:A\to \bigvee_\alpha SX_\alpha$ by $f(v_\alpha)=[(x_\alpha,1)]\forall\alpha$. It then follows that $SX\simeq\left(\bigvee_\alpha SX_\alpha\right)\sqcup_f G$, since we can collapse $G$ to identify every $[(x_\alpha,1)]$. Moreover, $f\simeq g$, where $g:A\to \bigvee_\alpha SX_\alpha$ maps every $v_\alpha$ to $[(x_\alpha,0)]$, and hence \[SX\simeq\left(\bigvee_\alpha SX_\alpha\right)\sqcup_f G\simeq\left(\bigvee_\alpha SX_\alpha\right)\sqcup_g G.\] $\left(\bigvee_\alpha SX_\alpha\right)\sqcup_g G$ is in turn homeomorphic to \[Y\bigvee_\alpha SX_\alpha,\] where $Y$ is a graph consisting of a single vertex and $l-1$ loops, where $l$ is the number of components of $X$.\newline

\noindent Now let $X$ be a finite graph. Let $T_\alpha$ be a spanning tree of $X_\alpha$, and let $n_\alpha$ be the number of edges in $X_\alpha$ that are not part of $T_\alpha$. $ST_\alpha$ is a contractible subcomplex, and every edge in $X_\alpha$ becomes a $2$-sphere in $SX_\alpha$. Hence, if $n_\alpha>0$, we have \[SX_\alpha\simeq S\left(\bigvee_{i=1}^{n_\alpha}S^1\right)\simeq\sum\left(\bigvee_{i=1}^{n_\alpha}S^1\right)=\bigvee_{i=1}^{n_\alpha}\sum S^1=\bigvee_{i=1}^{n_\alpha}S^2.\] Otherwise, if $n_\alpha=0$, then $SX_\alpha$ is simply contractible. If we define $\bigvee_{i=1}^0Q
$ to be a point for any space $Q$, we then have\[SX\simeq \bigvee_{i=1}^{l-1}S^1\bigvee_{i=1}^{\sum_\alpha n_\alpha}S^2,\] as required.
\end{proof}


\begin{exercise}
Use Corollary 0.20 to show that if $(X,A)$ has the homotopy extension property, then $X\times I$ deformation retracts to $X\times\{0\}\cup A\times I$. Deduce from this that Proposition 0.18 holds more generally for any pair $(X_1,A)$ satisfying the homotopy extension property.
\end{exercise}
\begin{proof}
Firstly, $(X\times I, X\times\{0\}\cup A\times I)$ also satisfies the homotopy extension property. Let \[\iota:X\times\{0\}\cup A\times I\to X\times I\] be the inclusion map. Define \[\phi:X\times I\to X\times\{0\}\cup A\times I:(x,s)\mapsto(x,0).\] Then \[\iota\circ \phi(x,s)=(x,0)\forall (x,s)\in X\times I.\] Define \[h_t:X\times I\to X\times I\] by \[h_t(x,s)=(x,ts).\] Then $h_0=\iota\circ\phi$ and $h_1=\text{id}_{X\times I}$, and hence \[\iota\circ\phi\simeq\text{id}_{X\times I}.\] Furthermore, \[\phi\circ\iota(x,s)=(x,0)\forall (x,s)\in X\times \{0\}\cup A\times I.\] Now define \[g_t:X\times\{0\}\cup A\times I\to X\times\{0\}\cup A\times I:(x,s)\mapsto(x,ts).\] Then $g_0=\phi\circ\iota$ and $g_1=\text{id}_{X\times\{0\}\cup A\times I}$, and hence \[\phi\circ\iota\simeq \text{id}_{X\times\{0\}\cup A\times I}.\] Corollary $0.20$ then implies that $X\times I$ deformation retracts to $X\times\{0\}\cup A\times I$.\newline


Now let $(X_1,A)$ be a pair satisfying the homotopy extension property, let $X_0$ be another topological space, and let $f,g:A\to X_0$ be homotopic. Let $F:A\times I\to X_0$ be a homotopy from $f$ to $g$. Consider the space $X_0\sqcup_F(X_1\times I)$. $X_1\times I$ deformation retracts onto $X_1\times\{0\}\cup A\times I$, and hence $X_0\sqcup_F(X_1\times I)$ deformation retracts onto $X_0\sqcup_F(X_1\times\{0\}\cup A\times I)$, which is in turn homeomorphic to $X_0\sqcup_fX_1$. Similarly, $X_1\times I$ deformation retracts onto $X_1\times\{1\}\cup A\times I$, and hence $X_0\sqcup_F(X_1\times I)$ deformation retracts onto $X_0\sqcup_gX_1$. Furthermore, these deformation retractions both fix $X_0$, and hence $X_0\sqcup_fX_1\simeq X_0\sqcup_gX_1\text{ rel }X_0$.
\end{proof}


\begin{exercise}
Given a pair $(X,A)$ and a homotopy equivalence $f:A\to B$, show that the natural map $X\to B\sqcup_fX$ is a homotopy equivalence if $(X,A)$ satisfies the homotopy extension property. An interesting case is when $f$ is a quotient map, hence the map $X\to B\sqcup_f X$ is the quotient map identifying each set $f^{-1}(b)$ to a point. When $B$ is a point this gives another proof of Proposition $0.17$.
\end{exercise}
\begin{proof}
By Corollary $0.21$, $A$ is a deformation retract of $M_f$, and hence $X$ is a deformation retract of $X\cup M_f$. By the previous exercise, $X\times\{0\}\cup A\times I$ is a deformation retract of $X\times I$, and hence $X\cup M_f$ is a deformation retract of $(X\times I)\sqcup B/\sim$, where $\sim$ is given by $(a,1)\sim f(a)\forall a\in A$. $(X\times I\sqcup B)/\sim$ in turn deformation retracts onto $(X\times\{1\}\sqcup B)/\sim$, which is homeomorphic to $B\sqcup_fX$.\[X\overset{\iota_1}{\longrightarrow} X\cup M_f\overset{\iota_2}{\longrightarrow} (X\times I\sqcup B)/\sim\overset{r}{\longrightarrow} B\sqcup_fX.\] $\iota_1:X\hookrightarrow X\cup M_f$ is a homotopy equivalence, $\iota_2:X\cup M_f\hookrightarrow (X\times I\sqcup B)/\sim$ is a homotopy equivalence, and the retraction $r:(X\times I\sqcup B)/\sim\to B\sqcup _fX$ given by $r([(x,s)])=[x]\forall (x,s)\in X\times I$ and $r([b])=[b]\forall b\in B$ is a homotopy equivalence. The composition of homotopy equivalences is a homotopy equivalence, and hence $r\circ \iota_2\circ\iota_1:X\to B\sqcup_fX$, which is precisely the natural map $X\to B\sqcup_fX$, is a homotopy equivalence.
\end{proof}

\begin{exercise}
Show that if $(X_1,A)$ satisfies the homotopy extension property, then so does every pair $(X_0\sqcup_fX_1,X_0)$ obtained by attaching $X_1$ to a space $X_0$ via a map $f:A\to X_0$.
\end{exercise}
\begin{proof}
Suppose we have a map $f_0:X_0\sqcup_fX_1\to Y$ and a homotopy $f_t:X_0\to Y$ of $f_0|X_0$. This in turn induces a homotopy on $A$ given by $f_t\circ f$. Since $(X_1,A)$ satisfies the HEP, it follows that $f_t\circ f:A\to Y$ can be extended to a homotopy $g_t:X_1\to Y$ with $g_0(x)=f_0([x])\forall x\in X_1$. We can thus extend $f_t$ to a homotopy $f_t:X_0\sqcup_fX_1\to Y$ given by \[f_t([x])=\begin{cases}
    g_t(x)&\text{if }x\in X_1,\\
    f_t(x)&\text{if }x\in X_0.
\end{cases}\]This is well-defined, for if $x\in A$, then $g_t(x)=f_t\circ f(x)=f_t([x])$. We now show that $f_t$ is continuous. First, define a homotopy \[F:(X_0\sqcup X_1)\times I\to Y:F(x,t)\mapsto f_t([x]).\] Let $\iota_0,\iota_1$ be the inclusion maps of $X_0$ and $X_1$ respectively into $X_0\sqcup X_1$. $\iota_0(X_0)\times I$ and $\iota_1(X_1)\times I$ are both closed in $(X_0\sqcup X_1)\times I$ as a consequence of the coproduct topology, \[\iota_0(X_0)\times I\cup \iota_1(X_1)\times I=(X_0\sqcup X_1)\times I,\] and $F$ is continuous when restricted to both $\iota_0(X_0)\times I$ and $\iota_1(X_1)\times I$, and hence $F$ is continuous on $(X_0\sqcup X_1)\times I$ by the Pasting lemma. Now let \[q:(X_0\sqcup X_1)\times I\to (X_0\sqcup_fX_1)\times I:(x,t)\mapsto([x],t)\] be the quotient map. $F$ then descends to a map \[G:(X_0\sqcup_fX_1)\times I\to Y:([x],t)\mapsto F(x,t),\] which is continuous by the universal property of the final topology, since $F=G\circ q$. Finally, \[G([x],t)=f_t([x])\forall[x]\in X_0\sqcup_fX_1,\] and hence $f_t$ is continuous, so a homotopy. Thus $(X_0\sqcup_fX_1,X_0)$ has the HEP. 
\newline Note - To justify that $q$ is actually a quotient map, we invoke the following result from "General Topology" by Engelking (theorem 3.3.17):\newline

\noindent Let $X$  be a locally compact space. Let $q:Y\to Z$ be a quotient map. Let the map \[f:X\times Y\to X\times Z\] be defined by $f(x,y)=(x,q(y))$ for each $(x,y)\in X\times Y$. Then the map $f$ is a quotient map from $X\times Y$ to $X\times Z$.

Indeed, $I$ is a compact Hausdorff space, and hence locally compact, so the result applies.
\end{proof}

\begin{exercise}
In case the CW complex $X$ is obtained from a subcomplex $A$ by attaching a single cell $e^n$, describe exactly what the extension of a homotopy $f_t:A\to Y$ to $X$ given by the proof of Proposition $0.16$ looks like. That is, for a point $x\in e^n$, describe the path $f_t(x)$ for the extended $f_t$.
\end{exercise}
\begin{proof}
Define $F:X\times I\to Y$ by $F(x,t)=f_t(x)$. The extension of $F:X\times\{0\}\cup A\times I\to Y$ is simply $F\circ R:X\times I\to Y$, where $R$ is the retraction of $X\times I$ to $X\times\{0\}\cup A\times I$ given in Proposition 0.16.

Let $r:D^n\times I\to D^n\times\{0\}\cup \partial D^n\times I$ be the retraction of $D^n\times I$ described in Proposition 0.16. 
We consider three different cases:
\begin{enumerate}
\item If $x$ is in the centre of $e^n$, then $r(x,t)=(x,0)\forall t\in I$. 
\item If $2-2\|x\|> 1$, then $r(x,t)$ simply moves $x$ radially outwards from the centre towards $\frac{x}{\|x\|}$ as $t$ increases, while keeping the $I$ component at $0$. In particular, if $p_x:I\to D^n:t\mapsto\frac{2x}{2-t}$ is the line $x$ travels on outwards from the centre as $t$ increases, then we have $r(x,t)=(p_x(t),0)$.
\item Otherwise,  $r(x,t)$ moves radially outwards from $(x,0)$ at $t=0$ to $\left(\frac{x}{\|x\|},0\right)$ at $t=2-2\|x\|$. For $t\geq2-2\|x\|$, we simply increase the value of the $I$ component. Explicitly, if we let $p_x:[0,2-2\|x\|]\to D^n:t\mapsto\frac{2x}{2-t}$ be the line from $x$ to $\frac{x}{\|x\|}$, then\[r(x,t)=\begin{cases}
    (p_x(t),0)&\text{if }t< 2-2\|x\|,\\
    \left(\frac{x}{\|x\|},2-\frac{2-t}{\|x\|}\right)&\text{if }t\geq2-2\|x\|.
\end{cases}\]
\end{enumerate}
We can now describe the path of $f_t(x)$. Let $\phi:\partial D^n\to A$ be the attaching map of $e^n$.
\begin{enumerate}
\item In the first situation, where $x$ is in the centre of $e^n$, then $f_t(x)=f_0(x)\forall t$.
\item In the second situation, where $2-2\|x\|> 1$, then $f_t(x)=f_0\left(\frac{2x}{2-t}\right)\forall t$.
\item In the final situation, where $2-2\|x\|\leq 1$, then\[f_t(x)=\begin{cases}
    f_0\left(\frac{2x}{2-t}\right)&\text{if }t<2-2\|x\|,\\
    f_{2-\frac{2-t}{\|x\|}}\circ\phi\left(\frac{x}{\|x\|}\right)&\text{if }t\geq 2-2\|x\|.
\end{cases}\]
\end{enumerate}
\end{proof}
\section{The Fundamental Group}
\subsection{Basic Constructions}
\begin{exercise}
Show that the composition of paths satisfies the following cancellation property: If $f_0\cdot g_0\sim f_1\cdot g_1$ and $g_0\sim g_1$ then $f_0\sim f_1$.
\end{exercise}
\begin{proof}
Let $\overline{g_0}$ and $\overline{g_1}$ be the reversals of $g_0$ and $g_1$ respectively. Then\[f_0\sim f_0\cdot g_0\cdot\overline{g_0}\sim f_1\cdot g_1\cdot \overline{g_0}\sim f_1\cdot g_1\cdot \overline{g_1}\sim f_1.\]
\end{proof}

\begin{exercise}
Show that the change-of-basepoint homomorphism $\beta_h$ depends only on the homotopy class of $h$.
\end{exercise}
\begin{proof}
Let $h\sim g$ as paths from $x_0$ to $x_1$. Let $[f]\in\pi_1(X,x_1)$. Then \[\beta_h([f])=[h\cdot f\cdot\overline{h}]=[g\cdot f\cdot\overline{h}]=[g\cdot f\cdot\overline{g}]=\beta_g([f]).\]
\end{proof}

\begin{exercise}
For a path-connected space $X$, show that $\pi_1(X)$ is abelian iff all basepoint-change homomorphisms $\beta_h$ depend only on the endpoints of the path $h$.
\end{exercise}
\begin{proof}
$(\implies):$ Let $h$ and $g$ be two paths from $x_0$ to $x_1$. Let $[f]\in\pi_1(X,x_1)$. Then\begin{align*}\beta_h([f])\beta_g([f])^{-1}&=[h\cdot f\cdot\overline{h}][g\cdot\overline{f}\cdot \overline{g}]\\&=[h\cdot f\cdot(\overline{h}\cdot g)\cdot\overline{f}\cdot \overline{g}]\\&=[h\cdot(\overline h\cdot g)\cdot f\cdot \overline{f}\cdot \overline{g}]\\&=e\end{align*}and hence $\beta_h([f])=\beta_g([f])$.

$(\impliedby):$ Let $[f],[g]\in\pi_1(X,x_1)$. We have $\beta_{[f]}=\beta_{[g]}$, so\[[f][g][f]^{-1}=[f\cdot g\cdot \overline{f}]=\beta_{[f]}([g])=\beta_{[g]}([g])=[g\cdot g\cdot \overline{g}]=[g]\] and hence\[[f][g]=[g][f].\] Thus $\pi_1(X)$ is abelian.
\end{proof}

\begin{exercise}
A subspace $X\subset\mathbb{R}^3$ is said to be star-shaped if there is a point $x_0\in X$ such that, for each $x\in X$, the line segment from $x_0$ to $x$ lies in $X$. Show that if a subspace $X\subset\mathbb{R}^3$ is locally star-shaped, in the sense that every point of $X$ has a star-shaped neighbourhood in $X$, then every path in $X$ is homotopic in $X$ to a piecewise linear path, that is, a path consisting of a finite number of straight line segments traversed at constant speed. Show this applies in particular when $X$ is open or when $X$ is a union of finitely many closed convex sets.
\end{exercise}
\begin{proof}
Let $p:I\to X$ be a path. For each $t\in I$, let $U_t$ be a star-shaped open neighbourhood of $p(t)$. By compactness (since the continuous image of a compact set is compact) there exists a finite subset $J\in I$ such that $\bigcup_{t\in J}U_t\cap p(I)$ is an open cover of $p(I)$. Suppose $J=\{t_1,...,t_n\}$, where $t_1<\dots<t_n$. Let $x_1,...x_n$ be points in $U_{t_1},...,U_{t_n}$ such that the line segment between every point $x\in U_{t_i}$ and $x_i$ is contained in $U_{t_i}$. Let $V_1,...,V_n$ denote $p^{-1}(U_{t_1}),...,p^{-1}(U_{t_n})$. $V_1,...,V_n$ then forms a finite open cover of $I$. By Lebesgue's number lemma, there exists a $\delta > 0$ such that every subset of $I$ with diameter less than $\delta$ is contained within some $V_i$. Let $N$ be some natural number such that $\frac{1}{N}<\delta$. We can then partition $I$ as $0,\frac{1}{N},\frac{2}{N},...,\frac{N-1}{N},1$. The diameter of $[\frac{i}{N},\frac{i+1}{N}]$ is less than $\delta$, and hence the interval is contained in some $V_j$ for every $i$. Let $k_i$ denote the index of a $U_{t_{j}}$ containing $p([\frac{i}{N},\frac{i+1}{N}])$. We define a map $q:I\to X$ as follows:

For each $i\in\{0,...,N-1\}$, we let $q(\frac{i}{N})=p(\frac{i}{N}),q(\frac{i+0.5}{N})=x_{k_i},q(\frac{i+1}{N})=p(\frac{i+1}{N})$. We then let $q$ trace out the line segment from $q(\frac{i}{N})$ to $q(\frac{i+0.5}{N})$ on the interval $[\frac{i}{N},\frac{i+0.5}{N}]$, and the line segment from $q(\frac{i+0.5}{N})$ to $q(\frac{i+1}{N})$ on the interval $[\frac{i+0.5}{N},\frac{i+1}{N}]$. We can then construct a homotopy from $p$ to $q$ as follows:

For each $i$, let $h_{i,t}:[\frac{i}{N},\frac{i+1}{N}]\to X$ be the straight-line homotopy from $p_{|[\frac{i}{N},\frac{i+1}{N}]}$ to $\gamma_{q(\frac{i+0.5}{N})}$ (note, this homotopy is well-defined by star-shapedness). Then let $a_{i,t}:[\frac{i}{N},\frac{i+1}{N}]\to X$ trace out the line segment from $q(\frac{i}{N})$ to $h_{i,t}(0)$, and let $b_{i,t}:[\frac{i}{N},\frac{i+1}{N}]\to X$ trace out the line segment from $h_{i,t}(1)$ to $q(\frac{i+1}{N})$. $a_{i,t}\cdot h_{i,t}\cdot b_{i,t}$ is then a based-homotopy from $p_{|[\frac{i}{N},\frac{i+1}{N}]}$ to $q_{|[\frac{i}{N},\frac{i+1}{N}]}$. Combining these then gives a based homotopy from $p$ to $q$. We can then reparametrize $q$ by arc-length to obtain the desired map.


When $X$ is open, every $x\in X$ is contained in a ball, which is star-shaped, and hence $X$ is locally star-shaped.

Now let $X$ be a union of finitely many closed convex sets $C_1,...,C_n$. Let $x\in X$. Let $C_{i_1},...,C_{i_m}$ be the collection of $C_i$'s which don't contain $x$. $X\setminus C_i$ is open $\forall C_i$, since $C_i$ is closed, and hence for each $C_{i_k}$ we can find an open ball $B_{i_k}\subseteq C_{i_k}^c$ which contains $x$. Consider $X\cap \bigcap_{k=1}^mB_{i_k}\subseteq X\setminus\bigcup_{k=1}^mC_{i_k}$. This set is open, as the finite intersection of open sets, and contains $x$. Let $y\in X\cap \bigcap_{k=1}^mB_{i_k}$. $y$ is then contained in a $C_i$ which also contains $x$. Since $C_i$ is convex, the line segment between $x$ and $y$ is then contained in $X$. Furthermore, the line segment between $x$ and $y$ is contained in $\bigcap_{k=1}^mB_{i_k}$, for the intersection of convex sets is convex, and hence the line segment between $x$ and $y$ is contained in $X\cap \bigcap_{k=1}^mB_{i_k}$. Hence $X\cap \bigcap_{k=1}^mB_{i_k}$ is a star-shaped neighbourhood of $x$, so $X$ is locally star-shaped.
\end{proof}

\begin{exercise}
Show that for a space $X$, the following three conditions are equivalent:
\begin{enumerate}
    \item[(a)] Every map $S^1\to X$ is homotopic to a constant map, with image a point.
    \item[(b)] Every map $S^1\to X$ extends to a map $D^2\to X$.
    \item[(c)] $\pi_1(X,x_0)=0\forall x_0\in X$.
\end{enumerate}
Deduce that a space $X$ is simply-connected iff all maps $S^1\to X$ are homotopic. [In this problem, 'homotopic' means 'homotopic without regard to basepoints'.]
\end{exercise}
\begin{proof}
$(a)\implies (b)$: Let $f:S^1\to X$ be a map homotopic to a constant map $\gamma_x$. We have a homotopy $H:S^1\times I\to X$ from $f$ to $\gamma_x$. Define the equivalence relation $\sim$ on $S^1\times I$ by $(\omega,1)\sim(\omega',1)\forall \omega,\omega'\in S^1$. We then obtain a well-defined map $g:S^1\times I/\sim\to X$ given by $g([\omega,t])=H(\omega,t)$. This is well-defined because $H(\omega,1)=x\forall \omega$. Let $\pi:S^1\times I\to S^1\times I/\sim$ be the quotient map. $g\circ\pi= H$ is continuous, and hence $g$ is continuous by the universal property of the quotient topology. Furthermore, $S^1\times I/\sim\cong D^2$. Let $\iota:S^1\hookrightarrow S^1\times I/\sim :\omega\mapsto[(\omega,0)]$ be the inclusion map. We then have that $f=g\circ \iota$, and hence $g$ is an extension of $f$ to $D^2$.

$(b)\implies (c)$: Let $p:I\to X$ be a loop. This then induces a map $p':S^1\to X$, after defining $S^1$ as the quotient space $I/0\sim 1$. We then have that $p'$ extends to a map $q:D^2\to X$, where we define $D^2$ as $S^1\times I/(\omega,1)\sim(\omega',1)$, which is in turn equal to $I^2/\sim$, where $\sim$ is the relation given by $(0,t)\sim(1,t),(t,1)\sim(s,1)\forall t,s\in I$. We now define a map $F:I^2\to X$ by $F((t,s))=q([(t,s)])$. Let $\pi:I^2\to I^2/\sim$ be the quotient map. We then have that $F=q\circ\pi$, and hence $F$ is continuous by the universal property of quotient spaces. Furthermore, $F(t,0)=p(t)$ and $F(t,1)$ is a constant map, and hence $p$ is homotopic to a constant map. However, while this is a homotopy of loops (since $F(0,s)=F(1,s)\forall s)$, the loops do not necessarily have the same basepoint as $p$. To resolve this, we define $h_t:I\to X:s\mapsto F(0,st)$, and let $\overline{h_t}$ be the reverse path of $h_t$. Let $f_t:I\to X$ be the homotopy from $p$ to the constant map given by $F$. We then have that $h_t\cdot f_t\cdot \overline{h_t}$ is a based-homotopy from $p$ to $h_1\cdot\overline{h_1}$, which is in turn based homotopic to $\gamma_{x_0}$. Hence $[p]=[\gamma_{x_0}]$. This holds for all $p$, and hence $\pi_1(X,x_0)=0$. $x_0$ is arbitrary, so this then holds for all $x_0\in X$.

$(c)\implies (a)$ Suppose we have a map $f:S^1\to X$. If we define $S^1$ as $I/0\sim 1$, this gives a path $p:I\to X:t\mapsto f([t])$, which is continuous by the universal property of the quotient topology. Let $x_0=p(0)$. Since $\pi_1(X,x_0)=0$, if follows that $p$ is based-homotopic to $\gamma_{x_0}$. Let $F:I^2\to X$ be the homotopy, so that $F(t,0)=p(t),F(t,1)=x_0,F(0,s)=F(1,s)=x_0$. If we identify $S^1\times I$ with $I^2/(0,t)\sim (1,t)$, we can then obtain a homotopy $G:S^1\times I\to X:[(t,s)]\mapsto F((t,s))$ from $f$ to a constant map, which is continuous by the universal property of the quotient topology, and which is well-defined, since $F(0,s)=F(1,s)\forall s$. Hence $f$ is homotopic to a constant map, with image a point.

Finally, suppose all maps $S^1\to X$ are homotopic. That in particular implies that every map is homotopic to a constant map, and hence that $\pi_1(X,x_0)=0\forall x_0\in X$ ($(a)\implies(c)$). Now let $x_0$ be some chosen point in $X$. We have that for every $x\in X$, the constant map $\gamma_x:S^1\to X:\omega\mapsto x$ is homotopic to $\gamma_{x_0}$. This implies that there is a path from $x$ to $x_0$. Hence $X$ is path-connected, so is simply-connected.

Now suppose that $X$ is simply connected. We pick some point $x_0$ and will show that every map $f:S^1\to X$ is homotopic to $\gamma_{x_0}$, since then all maps $S^1\to X$ will be homotopic due to homotopy being an equivalence relation. Given a map $f:S^1\to X$, we have that $f$ is homotopic to a constant map. Let $\gamma_x$ be that constant map. Since $X$ is simply-connected, and hence path-connected, there exists a path $p:I\to X$ from $x$ to $x_0$. We can then obtain a homotopy from $\gamma_x$ to $\gamma_{x_0}$ given by $h_t:S^1\to X:\omega\to p(t)$. It then follows by transitivity that $f$ is homotopic to $\gamma_{x_0}$, as required.

Hence $X$ is simply-connected iff all maps $S^1\to X$ are homotopic.
\end{proof}

\begin{exercise}
We can regard $\pi_1(X,x_0)$ as the set of basepoint-preserving homotopy classes of maps $(S^1,s_0)\to(X,x_0)$. Let $[S^1,X]$ be the set of homotopy classes of maps $S^1\to X$, with no conditions on basepoints. Thus there is a natural map $\Phi:\pi_1(X,x_0)\to [S^1,X]$ obtained by ignoring basepoints. Show that $\Phi$ is surjective if $X$ is path-connected, and that $\Phi([f])=\Phi([g])$ iff $[f]$ and $[g]$ are conjugate in $\pi_1(X,x_0)$. Hence $\Phi$ induces a bijection between $[S^1,X]$ and the set of conjugacy classes in $\pi_1(X)$, when $X$ is path-connected.
\end{exercise}
\begin{proof}
We will assume without loss of generality that $S^1:=I/0\sim 1$, and that $s_0=[0]$. Suppose that $X$ is path-connected. Let $f:S^1\to X$ be a representative of an element of $[S^1,X]$ such that $f(s_0)=x$, for some $x\in X$. Let $f':I\to X$ be the corresponding loop of $f$, so that $f([t])=f'(t)$. Let $p:I\to X$ be a path from $x_0$ to $x$. We can then construct a loop $g:S^1\to X\in\pi_1(X,x_0)$ by defining its corresponding map $g':I\to X$ by $g'=p\cdot f'\cdot\overline{p}$ (where each path is traversed at $3$ times the speed), which is then homotopic to $f'$ under the homotopy $F':I\times I\to X$ given by\[F'(t,s)=\begin{cases}
    p(s+3t(1-s))&\text{if }t\leq\frac{1}{3},\\
    f'(3(t-\frac{1}{3}))&\text{if }t\in[\frac{1}{3},\frac{2}{3}],\\
    \overline{p}(3(1-s)(t-\frac{2}{3}))&\text{if }t\geq\frac{2}{3}.
\end{cases}\]Since $F'(0,s)=F'(1,s)\forall s$, it follows that $F'$ also gives a well-defined homotopy $F:S^1\times I\to X$ from $g$ to $f$ (up to reparametrization) given by $F([t],s)=F'(t,s)$. Hence $\Phi([g])=[f]$ so $\Phi$ is surjective.

let $f,g:(S^1,s_0)\to (X,x_0)$ be loops such that $[f]$ and $[g]$ are conjugate; that is, $[f]=[a]^{-1}[g][a]$ for some loop $a:(S^1,s_0)\to (X,x_0)$. Let $f',g',a'$ be the corresponding maps of $I\to X$. We first observe that $g'$ is homotopic to $\overline{a}'\cdot g'\cdot a'$ when we don't require basepoints to be fixed. To see this, observe the map $H':I\times I\to X$ given by \[H'(t,s)=\begin{cases}
    \overline{a}'(1-s+3st)&\text{if }t\leq \frac{1}{3},\\
    g'(3(t-\frac{1}{3}))&\text{if }t\in[\frac{1}{3},\frac{2}{3}],\\
    a'(3s(t-\frac{2}{3}))&\text{if }t\geq\frac{2}{3}.
\end{cases}\] $H'(t,0)$ is a reparametrization of $g'$, and $H'(t,1)=\overline{a}'\cdot g'\cdot a'(t)$, and hence $H'$ is a homotopy from $g'$ to $\overline{a}'\cdot g'\cdot a'$ (up to reparametrization). It then follows from transitivity that $g'$ is homotopic to $f'$, and hence that $\Phi([g])=\Phi([f])$.

Now suppose $\Phi([f])=\Phi([g])$. Let $'G:I\times I\to X$ be a homotopy from $f'$ to $g'$, which does not necessarily preserve basepoints. Define $p:I\to X$ by $p(t)=G'(0,t)$. Consider the map $U':I\times I\to X$ given by\[U'(t,s)=\begin{cases}
    p(3st)&\text{if }t\leq\frac{1}{3},\\
    G'(3(t-\frac{1}{3}),s)&\text{if }t\in[\frac{1}{3},\frac{2}{3}],\\
    \overline{p}(1-s+3s(t-\frac{2}{3}))&\text{if }t\geq\frac{2}{3}.
\end{cases}\] This is a based homotopy between $f$ and $p\cdot g\cdot\overline{p}$ (up to reparametrization). Hence, $[f]$ and $[g]$ are conjugate in $\pi_1(X,x_0)$.

Let $\sim$ be the equivalence relation of conjugation on $\pi_1(X)$. and let $\Psi:\pi_1(X)/\sim\to[S^1,X]:[a]\mapsto\Phi(a)$ be the induced map on conjugacy classes. This map is well-defined, since if $[f]$ and $[g]$ are conjugate in $\pi_1(X)$, then $\Psi([[f]])=\Phi([f])=\Phi([g])=\Psi([[g]])$. Furthermore, $\Psi$ is injective, since if $\Psi([[f]])=\Psi([[g]])$, then $[[f]]=[[g]]$. Hence, if $X$ is path-connected, $\Psi$ is both injective and surjective, so is bijective.
\end{proof}

\begin{exercise}
Define $f:S^1\times I\to S^1\times I$ by $f(\theta,s)=(\theta+2\pi s,s)$, so $f$ restricts to the identity on the boundary circles of $S^1\times I$. Show that $f$ is homotopic to the identity by a homotopy $f_t$ that is stationary on one of the boundary circles, but not by any homotopy $f_t$ that is stationary on both boundary circles. [Consider what $f$ does to the path $s\mapsto(\theta_0,s)$ for fixed $\theta_0\in S^1$.]
\end{exercise}
\begin{proof}
Let $f_t$ be a homotopy from $\text{id}_{S^1\times I}$ to $f$ with the associated map $F$ defined as \[F:S^1\times I\times I\to S^1\times I:(\theta,s,t)\mapsto(\theta+2\pi st,s).\] When $s=0$, we have $f_t(\theta,0)=(\theta,0)\forall t$, as required.

Suppose for a contradiction that there exists a homotopy $f_t$ from $f$ to $\text{id}_{S^1\times I}$ which is stationary on both boundary circles. 
Consider the path $p:I\to S^1\times I:s\mapsto(\theta_0,s)$. We then have a homotopy of paths \[f_t\circ p:I\to S^1\times I\] such that \[f_0\circ p(s)=(\theta_0+2\pi s,s)\] and \[f_1\circ p(s)=(\theta_0,s).\] Let $\pi_{S^1}:S^1\times I\to S^1$ be the projection map from $S^1\times I$ to $S^1$. Then $\pi_{S^1}\circ f_t\circ p:I\to S^1$ is a homotopy from \[\pi_{S^1}\circ f_0\circ p:I\to S^1:s\mapsto\theta_0+2\pi s\] to \[\pi_{S^1}\circ f_1\circ p:I\to S^1:s\mapsto\theta_0.\] Since $f_t$ is stationary on both boundary circles, it follows that \[\pi_{S^1}\circ f_t\circ p(0)=\pi_{S^1}\circ f_t\circ p(1)=\theta_0\forall t,\] and hence $\pi_{S^1}\circ f_t\circ p$ is a homotopy of loops based at $\theta_0$. However, $\pi_{S^1}\circ f_0\circ p$ is not homotopic to the constant loop; a contradiction. Hence no such homotopy can exist.
\end{proof}

\begin{exercise}
Does the Borsuk-Ulam theorem hold for the torus? In other words, for every map $f:S^1\times S^1\to\mathbb{R}^2$ must there exist $(x,y)\in S^1\times S^1$ such that $f(x,y)=f(-x,-y)$?
\end{exercise}
\begin{proof}
Let $S^1$ be the unit circle in $\mathbb{R}^2$. We then define $f:S^1\times S^1\to\mathbb{R}^2$ by $f(x,y)=x$. Then \begin{align*}
    f(x,y)=f(-x,-y)&\iff x=-x\\&\iff x=(0,0).
\end{align*} However, $(0,0)\not\in S^1$; a contradiction.
\end{proof}

\begin{exercise}
Let $A_1,A_2,A_3$ be compact sets in $\mathbb{R}^3$. Use the Borsuk-Ulam theorem to show that there is one plane $P\subset\mathbb{R}^3$ that simultaneously divides each $A_t$ into two pieces of equal measure.
\end{exercise}
\begin{proof}
Let $\lambda_3:\mathcal{B}_3\to\mathbb{R}$ be the three-dimensional Lebesgue measure. $A_1,A_2,A_3$ are all measurable sets with finite Lesbesgue measure, due to compactness. Let $N:=(0,0,0,1)$, and let $x\in S^3$. Then let $K_x$ be the hyperplane passing through $N$ and normal to $x$. Then let $H_x$ be the intersection of $K_x$ with the hyperplane $x_4=0$, so that $H_x$ will be $\emptyset$ when $x=\pm N$, and a plane otherwise. Now let $B_x^+$ be the connected component of $\mathbb{R}^4\setminus K_x$ containing $x$, and let $B_x^-$ be the other connected component. We note that $B_{-x}^\pm=B_x^\mp$, since changing the sign of $x$ does not alter $K_x$, but it does alter which side of $K_x$ that $x$ lies on. Now we define $C_x^\pm:=B_x^\pm\cap\mathbb{R}^3$. We then observe that \[C_{-x}^+=B_{-x}^+\cap\mathbb{R}^3=B_x^-\cap\mathbb{R}^3=C_x^-.\] Now let \[f_i:S^3\to\mathbb{R}:x\mapsto\lambda_3(C_x^+\cap A_i).\] We then have \[f_i(-x)=\lambda_3(C_{-x}^+\cap A_i)=\lambda_3(C_x^-\cap A_i)\forall x\in S^3.\] We now define \[f:S^3\to\mathbb{R}^3:x\mapsto(f_1(x), f_2(x), f_3(x)),\] which is continuous by the universal property of the product topology. Hence, by the Borsuk-Ulam theorem, $\exists x_0\in S^3$ such that $f(x_0)=f(-x_0)$, or where $\lambda_3(C_{x_0}^+\cap A_i)=\lambda_3(C_{x_0}^-\cap A_i)\forall i$. This means that $H_{x_0}$ splits $A_i$ into two equal parts $\forall i$, as required.

\includegraphics[scale=0.5]{Screenshot (2078).png}
\end{proof}

\begin{exercise}
From the isomorphism $\pi_1(X\times Y,(x_0,y_0))\cong \pi_1(X,x_0)\times\pi_1(Y,y_0)$ it follows that loops in $X\times\{y_0\}$ and $\{x_0\}\times Y$ represent commuting elements of $\pi_1(X\times Y,(x_0,y_0))$. Construct an explicit homotopy demonstrating this.
\end{exercise}
\begin{proof}
Let $\alpha:I\to X\times\{y_0\}:t\mapsto(\alpha_X(t),y_0)$ and $\beta:I\to \{x_0\}\times Y:t\mapsto(x_0,\beta_Y(t))$ be loops, where $\alpha_X$ and $\beta_Y$ are loops in $X$ and $Y$ respectively. Then $\alpha\cdot\beta:I\to X\times Y$ is given by \[\alpha\cdot\beta(t)=\begin{cases}
    (\alpha_X(2t),y_0)&\text{if }t\in[0,\frac{1}{2}],\\(x_0,\beta_Y(2t-1))&\text{if }t\in[\frac{1}{2},1].
\end{cases}\]
and $\beta\cdot\alpha:I\to X\times Y$ is given by \[\beta\cdot\alpha(t)=\begin{cases}
    (x_0,\beta_Y(2t))&\text{if }t\in[0,\frac{1}{2}],\\(\alpha_X(2t-1),y_0)&\text{if }t\in[\frac{1}{2},1].
\end{cases}\] Define $F:I\times I\to X$ by \[F(s,t)=\begin{cases}
    \alpha_X(2(t-\frac{s}{2}))&\text{if }t\in[\frac{s}{2},\frac{s+1}{2}],\\
    x_0&\text{otherwise}.
\end{cases}\]and $G:I\times I\to Y$ by\[G(s,t)=\begin{cases}
    \beta_Y(2(t-\frac{1-s}{2}))&\text{if }t\in[\frac{1}{2}-\frac{s}{2},1-\frac{s}{2}],\\
    y_0&\text{otherwise}.
\end{cases}\]

We can then define $H:I\times I\to X\times Y:(s,t)\mapsto(F(s,t),G(s,t))$, which is a based-homotopy from $\alpha\cdot\beta$ to $\beta\cdot\alpha$.
\end{proof}

\begin{exercise}
If $X_0$ is the path-component of a space $X$ containing the basepoint $x_0$, show that the inclusion $X_0\hookrightarrow X$ induces an isomorphism $\pi_1(X_0,x_0)\to\pi_1(X,x_0)$.
\end{exercise}
\begin{proof}
Let $\iota_*:\pi_1(X_0,x_0)\to\pi_1(X,x_0):[\gamma]\mapsto[\iota\circ\gamma]$ be the push-forward homomorphism of the inclusion map.

Let $\rho:I\to X$ be a loop with base-point $x_0$ so that $[\rho]\in\pi_1(X,x_0)$. Since $X_0$ is the path-component containing $x_0$, it follows that the image of $\rho$ is contained in $X_0$. Hence, we have $\rho=\iota\circ\kappa$, where $\kappa$ is defined as $\kappa:I\to X_0:t\mapsto \rho(t)$, and hence $[\rho]=[\iota\circ\kappa]=\iota_*([\kappa])\in\iota_*(\pi_1(X_0,x_0))$. Hence $\iota_*$ is surjective.

Now let $\alpha,\beta:I\to X$ be based-homotopic loops based at $x_0$. As before, we have $\alpha=\iota\circ\alpha_{X_0}$, where $\alpha_{X_0}$ is given by $\alpha_{X_0}:I\to X_0:t\mapsto \alpha(t)$, and similarly $\beta=\iota\circ\beta_{X_0}$. Since any based-homotopy from $\alpha$ to $\beta$ will have an image lying entirely inside $X_0$, it follows that any such homotopy can have its image restricted to $X_0$, thereby becoming a based-homotopy from $\alpha_{X_0}$ to $\beta_{X_0}$, as required. Hence $\iota_*$ is also injective, and so is an isomorphism.
\end{proof}

\begin{exercise}
Show that every homomorphism $\pi_1(S^1)\to\pi_1(S^1)$ can be realized as the induced homomorphism $\phi_*$ of a map $\phi:S^1\to S^1$.
\end{exercise}
\begin{proof}
Let $\Phi:\pi_1(S^1,s_0)\to\pi_1(S^1,s_1)$ be the homomorphism. Define $\gamma_n:I\to S_1:t\mapsto e^{2\pi int}s_0$ and $\rho_n:I\to S_1:t\mapsto e^{2\pi int}s_1$. Then $\pi_1(S^1,s_0)=\{[\gamma_n]:n\in\mathbb{Z}\}$ and $\pi_1(S^1,s_1)=\{[\rho_n]:n\in\mathbb{Z}\}$.  Let $m$ be such that $\Phi([\gamma_1])=[\rho_m]$. Furthermore, let $\theta$ be the angle between $(s_0)^m$ and $s_1$, so that $s_1=e^{i\theta}(s_0)^m$. Then define $\phi$ by $\phi:S^1\to S^1:\omega\mapsto e^{i\theta}(\omega)^m$. We then have \[\phi\circ\gamma_n(t)=e^{i\theta}(e^{2\pi i nt}s_0)^m=e^{2\pi i mnt}s_1=\rho_{mn}(t)\forall t,n.\] Hence \[\phi_*([\gamma_n])=[\phi\circ\gamma_n]=[\rho_{mn}]=\Phi([\gamma_n])\forall n,\] and hence $\Phi=\phi_*$.
\end{proof}

\begin{exercise}
Given a space $X$ and a path-connected subspace $A$ containing the basepoint $x_0$, show that the map $\pi_1(A,x_0)\to\pi_1(X,x_0)$ induced by the inclusion $A\hookrightarrow X$ is surjective iff every path in $X$ with endpoints in $A$ is homotopic to a path in $A$.
\end{exercise}
\begin{proof}
$(\implies)$ Suppose that $\iota_*:\pi_1(A,x_0)\to\pi_1(X,x_0)$ is surjective, meaning that every loop in $X$ with base-point $x_0$ is based-homotopic to a loop in $A$ with base-point $x_0$. Let $p:I\to X$ be a path in $X$ with endpoints in $A$. Let $a_0$ and $a_1$ be the start and end-points of $p$ respectively. Since $A$ is path-connected, there exists a path $q:I\to A$ from $a_1$ to $a_0$. $p\cdot q:I\to X$ is then a loop based at $a_0$. Furthermore, $\pi_1(A,x_0)\cong\pi_1(A,a_0)$ and $\pi_1(X,x_0)\cong\pi_1(X,a_0)$, and so by surjectivity of $\iota_*$, there exists a loop $r:I\to A$ such that $p\cdot q\sim r$. Hence, $p\sim p\cdot q\cdot \overline{q}\sim r\cdot\overline{q}$, with $r\cdot\overline{q}$ being a path in $A$ from $a_0$ to $a_1$, as required.

$(\impliedby)$ Let $p:I\to X$ be a loop based at $x_0$. There then exists a loop $q:I\to A$ such that $q\sim p$; ie, $\iota_*([q])=[p]$. Hence, $\iota_*$ is surjective.
\end{proof}

\begin{exercise}
Show that the isomorphism $\pi_1(X\times Y)\cong\pi_1(X)\times\pi_1(Y)$ in Proposition $1.12$ is given by $[f]\mapsto(p_{1*}([f]),p_{2*}([f]))$, where $p_1$ and $p_2$ are the projections of $X\times Y$ onto its two factors.
\end{exercise}
\begin{proof}
The isomorphism presented in Proposition $1.12$ is $[f]\mapsto([g],[h])$, where $g$ and $h$ are such that $f(t)=(g(t),h(t))$. However, $g$ and $h$ are precisely $p_1\circ f$ and $p_2\circ f$ respectively, and hence the isomorphism can be written as $[f]\mapsto([p_1\circ f],[p_2\circ f])=(p_{1*}([f]),p_{2*}([f]))$, as required.
\end{proof}
\begin{exercise}
Given a map $f:X\to Y$ and a path $h:I\to X$ from $x_0$ to $x_1$, show that $f_*\beta_h=\beta_{fh}f_*$ in the diagram at the right. \begin{tikzcd}
\pi_1(X, x_1) \arrow[r, "\beta_h"] \arrow[d, "f_*"] & \pi_1(X, x_0) \arrow[d, "f_*"] \\
\pi_1(Y, f(x_1)) \arrow[r, "\beta_{fh}"]           & \pi_1(Y, f(x_0))
\end{tikzcd}
\end{exercise}
\begin{proof}
Recall that $\beta_h:\pi_1(X,x_1)\to\pi_1(X,x_0)$ is the change-of-basepoint map given by \[\beta_h([\alpha])=[h\cdot \alpha\cdot \overline{h}].\] Let $p:I\to X$ be a loop based at $x_1$. We need to show that $f(h\cdot p\cdot\overline{h})\sim (f\circ h)\cdot f(p)\cdot(\overline{f\circ h})$. Indeed,\begin{align*}
    f(h\cdot p\cdot\overline{h})&\sim f(h)\cdot f(p)\cdot f(\overline{h})\\&\sim(f\circ h)\cdot f(p)\cdot(f\circ\overline{h})\\&\sim(f\circ h)\cdot f(p)\cdot(\overline{f\circ h}).
\end{align*}
\end{proof}

\begin{exercise}
Show that there are no retractions $r:X\to A$ in the following cases:
\begin{enumerate}
    \item[(a)] $X=\mathbb{R}^3$ with $A$ any subspace homeomorphic to $S^1$.
    \item[(b)] $X=S^1\times D^2$ with $A$ its boundary torus $S^1\times S^1$.
    \item[(c)] $X=S^1\times D^2$ and $A$ the circle shown in the figure.
    \item[(d)] $X=D^2\vee D^2$ with $A$ its boundary $S^1\vee S^1$.
    \item[(e)] $X$ a disk with two points on its boundary identified and $A$ its boundary $S^1\vee S^1$.
    \item[(f)] $X$ the Mobius band and $A$ its boundary circle.
\end{enumerate}
\includegraphics[scale=0.5]{Screenshot 2026-01-24 at 01-55-18 AT.dvi - AT.pdf.png}
\end{exercise}
\begin{proof}
\begin{enumerate}
    \item[(a)] Suppose there exists a retraction $r:\mathbb{R}^3\to A$. Then by proposition $1.17$, the push forward of the inclusion map, $\iota_*:\pi_1(A,x_0)\to\pi_1(\mathbb{R}^3,x_0)$ is injective for any $x_0\in A$. However, $\pi_1(A,x_0)\cong\mathbb{Z}$, while $\pi_1(\mathbb{R}^3,x_0)=0$, so $\iota_*$ cannot be injective; a contradiction. hence there is no retraction $r:\mathbb{R}^3\to A$.
    \item[(b)] Suppose there exists a retraction $r:S^1\times D^2\to S^1\times S^1$. Then $\iota_*:\pi_1(S^1\times S^1)\to\pi_1(S^1\times D^2)$ is injective. However, we note that \[\pi_1(S^1\times D^2)\cong\pi_1(S^1)\times\pi_1(D^2)\cong\mathbb{Z},\] while \[\pi_1(S^1\times S^1)\cong\pi_1(S^1)\times\pi_1(S^1)\cong\mathbb{Z}^2.\] Hence, as before, it is not possible for $\iota_*$ to be injective.
    \item[(c)] If there were a retraction, then the kernel of $\iota_*:A\to S^1\times D^2$ would be trivial. However, consider the loop which generates $\pi_1(A)$ by traversing $A$ once. By inspection, it is clear that such a loop is null-homotopic in $X$, contradicting the triviality of $\ker\iota_*$. Hence, there is no retraction $X\to A$.
    \item[(d)] $X$ can be given a cell complex structure by beginning with a $0$-cell $e_1^0$, then attaching two $1$-cells to it, and then attaching two $2$-cells to the $1$-cells, as shown in the picture. The subcomplexes formed by $e_1^0\cup e_1^1\cup e_1^2$ and $e_1^0\cup e_2^1\cup e_2^2$ are contractible, and hence $X$ is homotopy equivalent to the space formed by collapsing both subcomplexes, which is a singleton set, implying that $X$ is simply-connected.
    
    \includegraphics[scale=0.5]{Screenshot (2083).png}
    
    However, $S^1\vee S^1$ is clearly not simply-connected, and hence $\iota_*:\pi_1(S^1\vee S^1)\to\pi_1(D^2\vee D^2)$ cannot be injective, meaning that there cannot be a retraction.
    \item[(e)] $X$ is homotopy equivalent to the drawing shown, since the subcomplex $e_1^0\cup e_2^0\cup e_3^1$ can be collapsed to give a space homeomorphic to $X$. On the other hand, if we instead collapse $e_1^0\cup e_2^0\cup e_1^1\cup e_2^1\cup e_1^2$, then we obtain a space homeomorphic to $S^1$, and hence $\pi_1(X)\cong\mathbb{Z}$.

    \includegraphics[scale=0.5]{Screenshot (2087).png}

    Now consider a loop $p:I\to S^1\vee S^1$ in $S^1\vee S^1$ which starts at $e_2^0$, then goes along $e_1^1$, and then reaches $e_1^0$, which is identified with $e_2^0$. This loop represents a non-trivial element of $\pi_1(S^1\vee S^1)$. However, clearly $\iota_*([p])=0$, and hence $\iota_*$ has a non-trivial kernel, so is not injective. Hence $S^1\vee S^1$ is not a retract of $X$.
    \item[(f)] Both the Mobius band and its boundary circle have fundamental groups isomorphic to $\mathbb{Z}$. Let $p:I\to A$ be the loop given by beginning at the orange point, then traversing along the blue line to $e_2^0=e_4^0$, and then traversing along the blue line to $e_3^0=e_1^0$, and finally traversing along the blue line to the orange point. $p$ is then a representative of a generator of $\pi_1(A)$. Let $C$ be the centre-circle of $X$, which we note is a deformation retract of $X$. Let $s:X\to C$ be a deformation retraction given by projecting a point on $X$ orthogonally onto $C$.
    \includegraphics[scale=0.4]{Screenshot (2088).png}
    
    $s$ then induces an isomorphism $s_*:\pi_1(X)\to\pi_1(C)$. Furthermore, we observe that $s\circ p$ is a loop which wraps around $C$ twice, and hence $s_*\circ \iota_*([p])=2x$, where $x$ is a generator of $\pi_1(C)$. Hence, since $s_*$ is a group isomorphism, $\iota_*([p])$ is neither $0$ nor a generator of $\pi_1(X)$, so is of the form $ky$, where $k$ is an integer greater than $1$  and $y$ is a generator of $\pi_1(X)$. However, we also have $r_*\circ\iota_*=\mathbf{1}_{\pi_1(A)}$, and hence $r_*(ky)=kr_*(y)=z$, where $z$ is a generator of $\pi_1(A)$. This is a contradiction, since a generator of a group isomorphic to $\mathbb{Z}$ is not $k$ times another element, where $k > 1$. Hence, $r$ cannot exist.
\end{enumerate}
\end{proof}

\begin{exercise}
Construct infinitely many nonhomotopic retractions $S^1\vee S^1\to S^1$.
\end{exercise}
\begin{proof}
Let $A$ and $B$ be the two circles comprising the wedge sum, and let $x$ be their point of intersection. We define retractions $r_n:S^1\vee S^1\to S^1,n\in\mathbb{N}$ as follows: Let $r_{n|A}=\text{id}_A$. Then let $r_{n|B}$ be a loop in $A$ based at $x$ which wraps around $A$ exactly $n$ times. Each $r_n$ is continuous by the pasting lemma, and hence is a retraction. Furthermore, if we let $p:\mathbb{R}\to A$ be the standard covering map of $A$ and let $\overline{r}_n:I\to S^1$ be the corresponding loop, then the endpoint of the lift of $\overline{r}_n$ is $n$, and hence the $\overline{r}_n$'s are pairwise nonhomotopic. Hence, the $r_n$'s are also pairwise nonhomotopic, since otherwise a homotopy between different $r_n$'s would induce a homotopy between the corresponding $\overline{r}_n$'s.
\end{proof}

\begin{exercise}
Using Lemma $1.15$, show that if a space $X$ is obtained from a path-connected subspace $A$ by attaching a cell $e^n$ with $n\geq 2$, then the inclusion $A\hookrightarrow X$ induces a surjection on $\pi_1$. Apply this to show:
\item[(a)] The wedge sum $S^1\vee S^2$ has fundamental group $\mathbb{Z}$.
\item[(b)] For a path-connected CW complex $X$ the inclusion map $X^1\hookrightarrow X$ of its $1$-skeleton induces a surjection $\pi_1(X^1)\to\pi_1(X)$. [For the case that $X$  has infinitely many cells, see Proposition $A.1$ in the Appendix.]
\end{exercise}
\begin{proof}
Let $x_0\in e^n$, and let $y$ be another point in $e^n$, distinct from $x_0$. Then let $A_1:=X\setminus\{y\}$, and let $A_2:= e^n$. Let $p:I\to X$ be a loop based at $x_0$. By Lemma $1.15$, there exists a product of loops $a_1\cdot a_2\cdots a_k$, where each $a_i$ is a loop contained in $A_1$ or $A_2$. Let $\iota:X\setminus\{y\}\hookrightarrow X$ be the inclusion map from $X\setminus\{y\}$ to $X$. Since $A_2$ is simply connected, we observe that \[a_1\cdot a_2\cdots a_{k-1}\cdot a_k\sim b_1\cdot b_2 \cdots b_{m-1}\cdot b_m,\] where each $b_i$ is a loop contained in $A_1$, and hence $\iota_*([b_1\cdot b_2 \cdots b_{m-1}\cdot b_m])=[p]$, implying that $\iota_*:\pi_1(X\setminus\{y\})\to \pi_1(X)$ is surjective. We furthermore note that $A$ is a deformation retract of $X\setminus\{y\}$, and hence the inclusion map $\kappa:A\hookrightarrow X\setminus\{y\}$ induces an isomorphism $\pi_1(A)\to\pi_1(X\setminus\{y\})$. Hence, by functoriality, it follows that the inclusion map from $A$ to $X$, given by $\iota\circ\kappa$, induces a surjective map $\iota_*\circ\kappa_*:\pi_1(A)\to\pi_1(X)$.
\item[(a)] In this case, $S^1\vee S^2$ is formed by attaching $e^2$ to $A:=S^1$. We thus have a surjection \[\mathbb{Z}\cong\pi_1(S^1)\to \pi_1(S^1\vee S^2),\] meaning that $\pi_1(S^1\vee S^2)$ is isomorphic to a subgroup of $\mathbb{Z}$. Since all non-trivial subgroups of $\mathbb{Z}$ are isomorphic to $\mathbb{Z}$, and since $S^1\vee S^2$ is clearly not simply-connected, it follows that $\pi_1(S^1\vee S^2)\cong\mathbb{Z}$.
\item[(b)] $X$ can be formed from $X^1$ by repeatedly attaching cells of dimension greater than or equal to $2$. Furthermore, at each point, the partial construction of $X$ from $X^1$ is path-connected. Hence the push-forward of the inclusion map $X^1\hookrightarrow X$ is the composition of many surjective maps, and hence it induces a surjection $\pi_1(X^1)\to\pi_1(X)$.
\end{proof}

\begin{exercise}
Show that if $X$ is a path-connected $1$-dimensional CW complex with basepoint $x_0$ a $0$-cell, then every loop in $X$ is homotopic to a loop consisting of a finite sequence of edges traversed monotonically. [See the proof of Lemma $1.15$. This exercise gives an elementary proof that $\pi_1(S^1)$ is cyclic generated by the standard loop winding once around the circle. The more difficult part of the calculation of $\pi_1(S^1)$ is therefore the fact that no iterate of this loop is nullhomotopic.]
\end{exercise}
\begin{proof}
Let $p:I\to X$ be a loop based at $x_0$. Let $t_1,\dots,t_n$, written in increasing order, be the times at which $p$ hits a $0$-cell after fully traversing the closure of a $1$-cell, and let $t_0:=0 $. The closure of the $1$-cell which $p$ fully traverses over the interval $[t_i,t_{i+1}]$ is a deformation retract of $p([t_i,t_{i+1}])$, and hence each $p_{|[t_i,t_{i+1}]}$ is homotopic to a path $a_i$ from $p(t_i)$ to $p(t_{i+1})$ which is the result of composing $p_{|[t_i,t_{i+1}]}$ with a deformation retraction. And by a similar argument, if $t_n\neq1$, then $p_{|[t_n,1]}\sim\gamma_{p(t_n)}$. This then implies $p\sim a_1\cdots a_n$. Furthermore, the closure of the $1$-cell is then either homeomorphic to a line segment, or to a circle. Let $b_i$ be the path which traverses the closure of the $1$-cell monotonically. In the first case, $a_i\sim b_i$ by the linear homotopy (or rather, the composition of a homeomorphism into the line segment, followed by the linear homotopy, followed by the inverse of the homeomorphism), and in the second case, $a_i\sim b_i$, since both represent the same element of the fundamental group. Hence, $p\sim a_0\cdots a_n\sim b_0\cdots b_n$, as required.
\end{proof}

\begin{exercise}
Suppose $f_t:X\to X$ is a homotopy such that $f_0$ and $f_1$ are each the identity map. Use Lemma $1.19$ to show that for any $x_0\in X$, the loop $f_t(x_0)$ represents an element of the center of $\pi_1(X,x_0)$. [One can interpret the result as saying that a loop represents an element of the center of $\pi_1(X)$ if it extends to a loop of maps $X\to X$.]
\end{exercise}
\begin{proof}
Let $x_0\in X$. Let $g_\#:\pi_1(X,x_0)\to\pi_1(X,x_0)$ be the change-of-basepoint isomorphism induced by the loop $f_t(x_0)$. It then follows from Lemma $1.19$ that $g_\#$ is the identity map, meaning that \[[f_t(x_0)]\cdot [p]\cdot[f_t(x_0)]^{-1}=[p]\forall[p]\in\pi_1(X,x_0),\] implying that $[f_t(x_0)]\in Z(\pi_1(X,x_0))$.
\end{proof}


\subsection{Van Kampen's Theorem}
\begin{theorem}
Van Kampen theorem sketch proof (the hard part)
\end{theorem}
\begin{proof}
Let $N$ be the group generated by all elements of the form $i_{\alpha\beta}(\omega)i_{\beta\alpha}(\omega)^{-1}$ for some $\omega\in\pi_1(A_\alpha\cap A_\beta)$ and let $Q:=\frac{*_\alpha\pi_1(A_\alpha)}{N}$. Let $\overline{\Phi}:Q\to\pi_1(X):aN\mapsto\Phi(a)$ be the induced map. We want to show that $\overline{\Phi}$ is injective, since then $N$ will be the kernel of $\Phi$. We define a factorization of $[f]\in\pi_1(X)$ as being a (not necessarily reduced) word $[f_1]\cdots[f_n]\in*_\alpha\pi_1(A_\alpha)$ such that $f_1\cdots f_n\sim f$. We say that two factorizations are equivalent if they are related by some sequence of the following moves:\begin{enumerate}
    \item If $[f_i]$ and $[f_{i+1}]$ are both in $\pi_1(A_\alpha)$, we can combine them to $[f_if_{i+1}]$.
    \item if $[f]\in\pi_1(A_\alpha\cap A_\beta)$, then we can view $[f]$ as being in either $\pi_1(A_\alpha)$ or $\pi_1(A_\beta)$.
\end{enumerate} Hence, showing that $\overline{\Phi}$ is injective then amounts to showing that any two factorizations of an element of $\pi_1(X)$ are equivalent. Let $[f_1]\cdots[f_n]$ and $[f'_1]\cdots[f'_m]$ be two factorizations of $[f]\in\pi_1(X)$. We then have a homotopy of loops $F:I\times I\to X$ from $f_1\cdots f_n$ to $f'_1\cdots f'_m$. By Lebesgue's number lemma, there exists $0<s_1<\dots<s_k=1$ and $0<t_1<\dots<t_j=1$ such that each $F([s_{i},s_{i+1}]\times[t_v,t_{v+1}])$ is entirely within some $A_\alpha$. We can then refine the partitions as necessary so that each $f_i$ and $f_u'$ exactly occupies some continuous blocks of $s_l$'s. We can then use the openness of the $A_\alpha$'s to shift the columns in the inner-rows slightly so that each vertex is part of at most $3$ rectangles. Since we assumed each $A_\alpha\cap A_\beta\cap A_\gamma$ were path-connected, and that $x_0$ was contained in each $A_\alpha$, there is then a path between the image of a vertex to $x_0$ for every vertex. We number the rectangles from bottom left to top right, and let $p_i$ be a path which starts on the level of the top edge of the rectangle $i$, then goes over it, then goes down its right edge, and then goes in a straight line to the other side of the square. Each $p_i$ can be made into a concatenation of loops in $A_\alpha$'s by at each vertex it passes, going to $x_0$, and then back again. If we can show that $p_{i+1}$ is equivalent to $p_i$, then we would have shown inductively that $f_1\cdots f_n$ is equivalent to $f'_1\cdots f'_m$. To do so, we note that $p_i$ is equivalent to the factorization given by switching the loops on the edges of $i+1$ to being loops in the $A_\alpha$ corresponding to the rectangle $i+1$. we can then homotope the part of $p_i$ on the edges of rectangle $i+1$ by a path homotopy (which is allowed due to changing the open set that part of $p_i$ is considered to be a loop in) to produce $p_{i+1}$. To illustrate how this works, consider going from $p_9$ to $p_{10}$. The loop along the right edge of $9$ can be made to be a loop in the open set corresponding to $10$ by equivalence, and similarly for the loop between the bottom right of $9$ and top right of $5$, and the loop from the top left of $6$ and bottom right of $10$. The concatenation of those $3$ loops can then be homotoped so that $p_9$ gets turned into $p_{10}$.

\includegraphics[scale=0.5]{Screenshot 2026-02-02 at 04-07-41 AT.dvi - AT.pdf.png}
\end{proof}
\begin{exercise}
Show that the free product $G*H$ of nontrivial groups $G$ and $H$ has trivial center, and that the only elements of $G*H$ of finite order are the conjugates of finite-order elements of $G$ and $H$.
\end{exercise}
\begin{proof}
Suppose for a contradiction that $G*H$ has non-trivial center. Let $z\in Z(G*H)$ be a reduced word of length $m > 0$. Then let $g\in G\setminus\{e_G\}$, and let $h\in H\setminus\{e_H\}$. We then check the following cases:
\begin{enumerate}
    \item If $z$ begins and ends with an element of $G$, then $h^{-1}zh$ will be a reduced word of length $m+2$, making $h^{-1}zh=z$ impossible.
    \item If $z$ begins and ends with an element of $H$, then this is impossible for analogous reasons.
    \item If $z$ begins with an element of $G$ and ends with an element of $H$, then either $g^{-1}zg$ begins with an element of $G$ after getting reduced, in which case $g^{-1}zg$ will have length $m+1$ whereas $z$ will have length $m$, or $g^{-1}zg$ will begin with an element of $H$ after reduction, and so cannot be the same word as $z$.
    \item If $z$ begins with an element of $H$ and ends with an element of $G$, then this is impossible for analogous reasons.
\end{enumerate}
Hence, $Z(G*H)$ must be trivial.

Let $z\in G*H$ be of the form $a^{-1}ga$, where $a\in G*H$ and $g\in G$, where $g$ has finite order $n$. We then have \[(a^{-1}ga)^n=\overset{n\text{ times}}{(a^{-1}ga)\cdots (a^{-1}ga)}=a^{-1}g(aa^{-1})g(aa^{-1})\cdots ga=a^{-1}g^na=e,\] and hence all the conjugates of finite-order elements of $G$ have finite order. Similarly, all conjugates of finite-order elements of $H$ have finite order.

We now prove that these are the only elements of finite order. Let $z\in G*H$ be an element of finite order. That is, $\exists n\in\mathbb{N}$ such that $z^n$ is the empty word. Let $z=a_1\cdots a_m$ be in reduced form, where the $a_i$'s are letters. In order for $z^n=e$, we need for $a_i=a_{m-i+1}^{-1}\forall i \leq\frac{m}{2}$. If $m$ is even, this then means that $z$ is the empty word, which is the conjugate of $e_G$ with any element of $G*H$. If $m$ is odd, $z$ is then of the form $a^{-1}ba$, where $a\in G*H$ and $b$ is a letter such that $b^n$ is the empty-word, implying that $b$ is an element of $G$ or $H$ of finite order.
\end{proof}

\begin{exercise}
Let $X\subset \mathbb{R}^m$ be the union of convex open sets $X_1,\dots,X_n$ such that $X_i\cap X_j\cap X_k\neq\emptyset$ for all $i,j,k$. Show that $X$ is simply-connected.
\end{exercise}
\begin{proof}
We first note that each $X_i$ is path-connected, due to being a convex set. We will then prove the claim by induction. For $n=1$, the claim is trivial, since convex sets are always simply-connected. Now assume the claim is true for $n=k$; that is, that $\bigcup_{i=1}^kX_i$ is simply-connected. Now let $n=k+1$. We first note that $\bigcup_{i=1}^kX_i$ is open, being the union of open sets, and is furthermore path-connected, due to being simply-connected. We furthermore observe that $(\bigcup_{i=1}^kX_i)\cap X_{k+1}\supseteq X_1\cap X_{k+1}$ is non-empty. Finally, let $a,b\in (\bigcup_{i=1}^kX_i)\cap X_{k+1}=\bigcup_{i=1}^k(X_i\cap X_{k+1})$. Suppose that $a\in X_i\cap X_{k+1}$ and $b\in X_j\cap X_{k+1}$. Let $y\in X_i\cap X_j\cap X_{k+1}$, which exists by assumption. We can trace a path from $a$ to $y$, since the intersection of two convex sets is path-connected, and then we can similarly trace a path from $y$ to $b$, thereby giving a path from $a$ to $b$. Hence $(\bigcup_{i=1}^kX_i)\cap X_{k+1}$ is path-connected, so we can apply the Van Kampen theorem to conclude that the map \[\Phi:\pi_1\left(\bigcup_{i=1}^kX_i\right)*\pi_1(X_{k+1})\to\pi_1\left(\bigcup_{i=1}^{k+1}X_i\right)\] is surjective. Moreover, $\pi_1(\bigcup_{i=1}^kX_i)=0$ by the inductive hypothesis, and $\pi_1(X_{k+1})=0$ since convex sets are simply connected, and hence the domain of $\Phi$ is the trivial group, implying that $\pi_1(\bigcup_{i=1}^{k+1}X_i)$ must also be the trivial group due to surjectivity. Hence $\bigcup_{i=1}^{k+1}X_i$ is simply-connected, thus completing the induction. Hence, $X$ is simply-connected.
\end{proof}

\begin{exercise}
Show that the complement of a finite set of points in $\mathbb{R}^n$ is simply-connected if $n\geq 3$.
\end{exercise}
\begin{proof}
We prove this by induction. Let $x_1,\dots,x_m$ be the points removed, and let $X:=\mathbb{R}^n\setminus\{x_1,\dots,x_m\}$. For the base-case of one point $x_1$ removed, we note that $\mathbb{R}^n\setminus\{x_1\}$ is homotopy equivalent to $S^{n-1}$, and hence $X$ is simply-connected. Now assume the claim holds for $m=k$. That is, given any set of points $x_1,\dots,x_k$, then $\mathbb{R}^n\setminus\{x_1,\dots,x_k\}$ is simply-connected. Now let $m=k+1$. Assume without loss of generality that $x_{k+1}$ is the "furthest out" point, meaning that there exist open sets $\mathcal{U},\mathcal{V}$ such that $\mathcal{U}\cap\mathcal{V}$ is path-connected, $\mathcal{U}\cup\mathcal{V}=\mathbb{R}^n$, $\mathcal{U}\cong\mathcal V\cong\mathbb{R}^n$, and $\mathcal{U}$ contains $x_1\dots,x_k$ but not $x_{k+1}$, while $\mathcal{V}$ contains $x_{k+1}$ but not any of $x_1,\dots,x_k$  (we can think of these sets as the result of cutting $\mathbb{R}^n$ by a hyperplane, and then expanding the two fragments along the hyperplane uniformly). Then let $U:=X\cap\mathcal{U}$ and let $V:=X\cap\mathcal{V}$. Both $U$ and $V$ are path-connected open sets, $X=U\cup V$ and $U\cap V$ is path-connected. The Van Kampen theorem then applies to give a surjective homomorphism $\Phi:\pi_1(U)*\pi_1(V)\to\pi_1(X)$. We then note that $V$ is homotopy-equivalent to $S^{n-1}$, and hence, since $n\geq 3$, $\pi_1(V)$ is trivial. Furthermore, the inductive hypothesis implies that $\pi_1(U)$ is trivial. Hence we have a surjective homomorphism from the trivial group into $\pi_1(X)$, implying that $X$ is simply-connected.

Note: we can prove the existence of $\mathcal{U}$ and $\mathcal{V}$ as follows: Let $v$ be a unit vector such that each $v\cdot x_i$ is distinct. Such a $v$ exists, since otherwise $\mathbb{R}^n$ would be the union of a finite number of hyperplanes (as $v\cdot x_i=v\cdot x_j$ iff $x_i$ and $x_j$ lie in a hyperplane orthogonal to $v$), which is impossible, since if $\lambda_n$ is the $n$-dimensional Lebesgue measure, then $\lambda_n(\mathbb{R}^n)=\infty$, while $\lambda_n(\mathbb{R}^{n-1})=0$. We then cut along the hyperplane orthogonal to the point equidistant between $v\cdot x_a$ and $v\cdot x_b$, where $x_a$ and $x_b$ are the two furthest-out points.
\end{proof}

\begin{exercise}
Let $X\subset\mathbb{R}^3$ be the union of $n$ lines through the origin. Compute $\pi_1(\mathbb{R}^3\setminus X)$.
\end{exercise}
\begin{proof}
We first observe that $\mathbb{R}^3\setminus X$ is homotopy-equivalent to a $2$-sphere with $2n$ points removed. Indeed, the map $r_t:\mathbb{R}^3\setminus X\to \mathbb{R}^3\setminus X:(1-t)x+t\frac{x}{\|x\|}$ is a deformation retraction from $\mathbb{R}^3\setminus X$ onto the $2$-sphere with $2n$ points removed. Let $x_1,\dots,x_{2n}$ be distinct points on $S^2$. We then have $\pi_1(\mathbb{R}^3\setminus X)\cong\pi_1(S^2\setminus\{x_1,\dots,x_{2n}\})$. Furthermore, we observe that $S^2\setminus\{x_1,\dots,x_{2n}\}$ is homotopy equivalent to $D^2$ with $2n-1$ holes in its interior, and hence $\pi_1(\mathbb{R}^3\setminus X)\cong\pi_1(D^2\setminus\{y_1,\dots,y_{2n-1}\})$, where $y_1,\dots, y_{2n-1}$ are distinct points in the interior of $D^2$. We shall prove by induction that $\pi_1(D^2\setminus\{y_1,\dots,y_m\})\cong *_{i=1}^m\mathbb{Z}$. For the base case of $m=1$, we observe that the disc with a point removed is homotopy equivalent to a circle, and hence has fundamental group $\mathbb{Z}$. Now assume the claim holds for $m=k$. That is, $\pi_1(D^2\setminus\{y_1,\dots,y_k\})\cong *_{i=1}^k\mathbb{Z}$. Let $m=k+1$. Assume without loss of generality that $y_{k+1}$ is an "outer" point of the disc, meaning that we can slice the disc into two pieces, such that one piece contains the holes formed by $y_1,\dots, x_y$, and the other piece contains the hole formed by $y_{k+1}$. Let $U$ and $V$ be the first and second sections of the disc, plus a bit extra, so that both are open and path-connected, as well has having non-empty and simply-connected intersection. We can then apply the Van Kampen theorem to observe that the homomorphism $\Phi:\pi_1(U)*\pi_1(V)\to D^2\setminus\{y_1,\dots,y_{k+1}\}$ is surjective. Moreover, the inductive hypothesis implies that $\pi_1(U)\cong *_{i=1}^k\mathbb{Z}$, while $V$ is homotopy-equivalent to a circle, and hence $\pi_1(V)\cong\mathbb{Z}$. We then have a surjection $*_{i=1}^{k+1}\mathbb{Z}\to D^2\setminus\{y_1,\dots,y_{k+1}\}$. Finally, the Van Kampen theorem gives that the kernel of the map is the normal subgroup $N$ given by all elements of the form $i_{U}(\omega)i_V(\omega)^{-1}$ for $\omega\in\pi_1(U\cap V)$. However, $U\cap V$ is simply-connected, and hence $N$ is trivial, implying that $\Phi$ is an isomorphism. This completes the induction. We can then conclude that $\pi_1(\mathbb{R}^3\setminus X)\cong *_{i=1}^{2n-1}\mathbb{Z}$, the free group of rank $2n-1$.
\end{proof}

\begin{exercise}
Let $X\subset\mathbb{R}^2$ be a connected graph that is the union of a finite number of straight line segments. Show that $\pi_1(X)$ is free with basis consisting of loops formed by the boundaries of the bounded complementary regions of $X$, joined to a basepoint by suitably chosen paths in $X$. [Assume the Jordan curve theorem for polygonal simple closed curves, which is equivalent to the case that $X$ is homeomorphic to $S^1$.]
\end{exercise}
\begin{proof}

\end{proof}

\begin{exercise}
Use Proposition $1.26$ to show that the complement of a closed discrete subspace of $\mathbb{R}^n$ is simply-connected if $n\geq 3$.
\end{exercise}
\begin{proof}
Let $x_i$ be the points comprising the closed discrete subspace $C\subset\mathbb{R}^n$. Each $x_i$ has an $n$-ball $D_i^n$ containing it, such that $D_i^n$ does not contain any other points in $C$. Each $D_i^n\setminus\{x_i\}$ then deformation retracts onto an $(n-1)$-sphere $S_i^{n-1}$, and hence $\mathbb{R}^n\setminus C$ is homotopy equivalent to $\mathbb{R}^n\setminus\bigcup_iD_i^{n}$. Furthermore, $\mathbb{R}^n$ is obtained by attaching $n$-cells to $\mathbb{R}^n\setminus\bigcup_iD_i^{n}$, and so Proposition $1.26$ implies that $\pi_1(\mathbb{R}^n)\cong\pi_1(\mathbb{R}^n\setminus\bigcup_iD_i^{n})$, implying that $\mathbb{R}^n\setminus\bigcup_iD_i^{n}$, and hence $\mathbb{R}^n\setminus C$, is simply-connected.
\end{proof}

\begin{exercise}
Let $X$ be the quotient space of $S^2$ obtained by identifying the north and south poles to a single point. Put a cell complex structure on $X$ and use this to compute $\pi_1(X)$.
\end{exercise}
\begin{proof}
Consider the cell complex shown below, which we shall denote as $Z$. $X$ is then homeomorphic to $Z/A$, where $A$ is the subcomplex $e_0^0\cup e_1^1\cup e_1^0$. $A$ is contractible, and hence $X$ is homotopy equivalent to $Z$.

\includegraphics[scale=0.5]{Screenshot (2187).png}

\noindent Now let $X^1$ be the $1$-skeleton of $X$, so that $X$ is obtained from $X^1$ by attaching two $2$-cells along $Y\setminus e_1^1$. Proposition $1.26$ then implies that $\pi_1(X,e_0^0)\cong\pi_1(X^1,e_0^0)/N$, where $N$ is the normal subgroup generated by the loop $\phi:S^1\to X^1$, where $\phi$ traverses from $e_0^0$ along $e_0^1$ to $e_1^0$ and then to $e_0^0$ along $e_2^1$ at constant speed. $X^1$ is homotopy equivalent to $S^1\vee S^1$, after contracting $A$, and hence $\pi_1(X^1,e_0^0)\cong \mathbb{Z}*\mathbb{Z}$. If we let $a,b$ be the standard generators of $\pi_1(X^1,e_0^0)$, where $a$ traverses along the semicircle containing $e_0^1$ and $b$ traverses along the semicircle containing $e_2^1$, then $\phi=ab$, and hence $\pi_1(X)\cong\langle a,b\mid ab=e\rangle\cong\mathbb{Z}$.
\end{proof}

\begin{exercise}
Compute the fundamental group of the space obtained from two tori $S^1\times S^1$ by identifying a circle $S^1\times\{x_0\}$ in one torus with the corresponding circle $S^1\times\{x_0\}$ in the other torus.
\end{exercise}
\begin{proof}
Call the space $X$. We can construct a cell complex structure of $X$ as follows:

\includegraphics[scale=0.5]{Screenshot (2208)-1.png}

\noindent From this we observe that $X^1$ is a wedge sum of $3$ circles, as shown below:

\includegraphics[scale=0.5]{Screenshot (2208)-2.png}

\noindent By Proposition $1.26$, we have that $\pi_1(X)\cong\pi_1(X^1)/N$, where $N$ is the normal subgroup generated by the loops in $X^1$ corresponding to the attaching maps of the two $2$-cells. If we define the loops $a$, $b$ and $c$ as in the diagram, we then have both that $\pi_1(X^1)=\langle a,b,c\rangle$ and $N=\langle\langle aba^{-1}b^{-1}, aca^{-1}c^{-1}\rangle\rangle$, implying that $\pi_1(X)\cong\langle a,b,c\mid ab=ba,ac=ca\rangle$.
\end{proof}
\begin{exercise}
In the surface $M_g$ of genus $g$, let $C$ be a circle that separates $M_g$ into two compact subsurfaces $M_h'$ and $M_k'$ obtained from the closed surfaces $M_h$ and $M_k$ by deleting an open disk from each. Show that $M_h'$ does not retract onto its boundary circle $C$, and hence $M_g$ does not retract onto $C$. [Hint: abelianize $\pi_1$.] But show that $M_g$ does retract onto the nonseparating circle $C'$ in the figure.

\includegraphics[scale=0.5]{Screenshot 2026-02-10 at 21-57-39 AT.dvi - AT.pdf.png}
\end{exercise}
\begin{proof}
Suppose a retraction $r:C\to M_h'$ exists. Let $\iota:C\hookrightarrow M_h'$ be the inclusion map. Then we have $r\circ \iota=\text{id}_C$, implying, by functoriality, that $r_*\circ\iota_*=\text{id}_{\pi_1(C)}$. We can furthermore apply the abelianization functor to obtain maps \[r_*^\text{ab}:\pi_1(M_h')^\text{ab}\to\pi_1(C)^\text{ab}:[\gamma][\pi_1(M_h'),\pi_1(M_h')]\mapsto r_*([\gamma])[\pi_1(C),\pi_1(C)]\] and \[\iota_*^\text{ab}:\pi_1(C)^\text{ab}\to\pi_1(M_h')^\text{ab}:[\gamma][\pi_1(C),\pi_1(C)]\mapsto\iota_*([\gamma])[\pi_1(M_h'),\pi_1(M_h')].\] Functoriality then gives $r_*^{\text{ab}}\circ\iota_*^{\text{ab}}=(r_*\circ\iota_*)^\text{ab}=\text{id}_{\pi_1(C)^\text{ab}}$. Hence $\iota_*^{\text{ab}}$ has a left-inverse, implying that it is injective. We can construct $M_h'$ as a quotient of an $(4h+1)$-gon, which is the standard $4h$-gon construction of a genus $h$ surface, along with an extra edge which doesn't get identified with any other edge. As an example, the diagram below is the pentagon for $M_1'$.

\includegraphics[scale=0.3]{Screenshot (2249).png}

\noindent Call the complex $X$. The $1$-skeleton of this complex is a wedge-sum of $2h+1$ circles, so \[\pi_1(X^1)=\langle a_1,b_1,\dots,a_h,b_h,c\rangle.\] The attaching map of the $2$-cell is then given by $c[a_1,b_1]\cdots[a_h,b_h]$, and hence \begin{align*}\pi_1(M_h')&\cong\langle a_1,b_1,\dots,a_h,b_h,c\rangle/\langle\langle c[a_1,b_1]\cdots[a_h,b_h]\rangle\rangle\\&=\langle a_1,b_1,\dots,a_h,b_h,c\mid c=[a_h,b_h]^{-1}\cdots[a_1,b_1]^{-1}\rangle\\&\cong F_{2h}.\end{align*} Hence, if we take the abelianization, we have $\pi_1(M_h')^\text{ab}\cong F_{2h}^\text{ab}\cong\prod_{i=1}^{2h}\mathbb{Z}$. Furthermore, we have that \[\iota_*^\text{ab}(c[\pi_1(C),\pi_1(C)])=[a_h,b_h]^{-1}\cdots[a_1,b_1]^{-1}[\pi_1(M_h'),\pi_1(M_h')]=e_{\pi_1(M_h')^\text{ab}},\] implying that $\iota_*^\text{ab}$ is the trivial homomorphism. However, $\iota_*^\text{ab}$ is also injective; a contradiction. Hence, $C$ is not a retract of $M_h'$. Hence, $M_g$ does not retract onto $C$, since if there were a retraction $r:M_g\to C$, it would then restrict to a retraction $r_{|M_h'}:M_h'\to C$.
\end{proof}
\begin{exercise}
Consider two arcs $\alpha$ and $\beta$ embedded in $D^2\times I$ as shown in the figure. The loop $\gamma$ is obviously nullhomotopic in $D^2\times I$, but show that there is no nullhomotopy of $\gamma$ in the complement of $\alpha\cup\beta$.

\includegraphics[scale=0.5]{Screenshot 2026-02-18 at 00-08-45 AT.dvi - AT.pdf.png}
\end{exercise}

\begin{exercise}
The mapping torus $T_f$ of a map $f:X\to X$ is the quotient of $X\times I$ obtained by identifying each point $(x,0)$ with $(f(x),1)$. In the case $X=S^1\vee S^1$ with $f$ basepoint-preserving, compute a presentation for $\pi_1(T_f)$ in terms of the induced map $f_*:\pi_1(X)\to\pi_1(X)$. Do the same when $X=S^1\times S^1$. [One way to do this is to regard $T_f$ as built from $X\vee S^1$ by attaching cells.]
\end{exercise}
\begin{proof}
First let $X=S^1\vee S^1$. We can view $T_f$ as a cell complex by beginning with the $1$-skeleton shown below, and attaching a two $2$-cells by the attaching maps $\phi_0=acx^{-1}c^{-1}$ and $\phi_1=bcy^{-1}c^{-1}$. To obtain $T_f$, we then perform the identification $(x,0)\sim(f(x),1)\forall x\in X$, which results in $f_*(a)=x$ and $f_*(b)=y$. Hence, we can view $T_f$ as $S^1\vee S^1\vee S^1$, with the attaching maps $\psi_1=acf_*(a)^{-1}c^{-1}$ and $\psi_2=bcf_*(b)^{-1}c^{-1}$. Hence, $\pi_1(T_f)\cong\langle a,b,c\mid c^{-1}ac=f_*(a),c^{-1}bc=f_*(b)\rangle$.
\includegraphics[scale=0.3]{Screenshot (2255).png} 
\end{proof}




\begin{exercise}
Show that the subspace of $\mathbb{R}^3$ that is the union of the spheres of radius $1/n$ and center $(1/n,0,0)$ for $n=1,2,\dots$ is simply-connected.
\end{exercise}
\begin{proof}
Let $X$ be the subspace, and let $A_n$ be the sphere of radius $1/n$. Then let $B_n=\bigcup_{i=1}^n A_n$. Let $p:I\to X$ be a loop based at $(0,0,0)$. Let $r_n:X\to X$ be a retraction onto $A_n$ such that $r_n(X\setminus A_n)=\{(0,0,0)\}$. By Proposition $1.26$, each $A_n$ is semilocally simply-connected, and hence for each $n\in\mathbb{N}$, there exists a homotopy $R_n:[1-\frac{1}{n+1},1-\frac{1}{n+2}]\times I\to A_n$ relative to $p^{-1}(\{(0,0,0)\})$ from $r_n\circ p$ to $\gamma_{(0,0,0)}$. We then define $H_n:[1-\frac{1}{n+1},1-\frac{1}{n+2}]\times I\to X$ by\[H_1(s,t)=\begin{cases}
    R_1(s,t)&\text{if }p(t)\in A_1,\\p(t)&\text{if }p(t)\in\bigcup_{i=2}^\infty A_i
\end{cases}\]and\[H_n(s,t)=\begin{cases}
    H_{n-1}(1-\frac{1}{n+1},t)&\text{if }p(t)\in \bigcup_{i=1}^{n-1} A_i,\\R_n(s,t)&\text{if }p(t)\in A_n,\\p(t)&\text{if }p(t)\in\bigcup_{i=n+1}^\infty A_i
\end{cases}\]if $n>1$. $H_1$ is well-defined, since both $p(t)$ and $R_1(s,t)$ are equal to $(0,0,0)$ on $p^{-1}(\{(0,0,0)\})=A_n\cap(\bigcup_{i=2}^\infty A_i)$. And $H_n$ is well-defined $\forall n>1$, since the intersection of any pairs of $\bigcup_{i=1}^{n-1}A_i,A_n$ and $\bigcup_{i=n+1}^\infty A_i$ is $p^{-1}(\{(0,0,0\})$, and $H_{n-1}(1-\frac{1}{n+1},t),R_n(s,t)$ and $p(t)$ are all equal to $(0,0,0)$ on that set. Furthermore, both $A_1$ and $\bigcup_{i=2}^\infty A_i$ are closed, so $H_1$ is continuous by the pasting lemma, and $\bigcup_{i=1}^{n-1}A_i,A_n$ and $\bigcup_{i=n+1}^\infty A_i$ are closed, and hence $H_n$ is continuous by the pasting lemma for $n>1$.

We then define $F:I\times I\to X$ by\[F(s,t)=\begin{cases}
    p(t)&\text{if }s\in[0,\frac{1}{2}],\\H_1(s,t)&\text{if }s\in[\frac{1}{2},\frac{2}{3}],\\H_2(s,t)&\text{if }s\in[\frac{2}{3},\frac{3}{4}],\\\vdots&\vdots\\H_n(s,t)&\text{if }s\in[1-\frac{1}{n+1},1-\frac{1}{n+2}],\\\vdots&\vdots
\end{cases}\]This is continuous on $[0,1-\frac{1}{n+2}]\times I\forall n$ by the pasting lemma. Furthermore, $F(1,t)=(0,0,0)\forall t$, since if $p(t)\in A_n$ for some $n$, then $F(s,t)=(0,0,0)\forall s>\frac{1}{n+5}$. We now need to show that it is continuous on $\{1\}\times I$. Indeed, let $t\in I$ and let $\epsilon > 0$. Let $n\in\mathbb{N}$ be such that $\frac{2}{n}<\epsilon$. Then for all $(s',t')\in B_{\frac{1}{n+3}}((1,t))\cap (I\times I)$, we have \[F(s',t')\in\bigcup_{i=n+1}^\infty A_i\subseteq B_{\frac2{n+1}}((0,0,0))\subseteq B_{\epsilon}((0,0,0)),\] as required. We then have a based homotopy from $p$ to $\gamma_{(0,0,0)}$. Since $p$ is arbitrary, it follows that $X$ is simply-connected.
\end{proof}
\end{document}
