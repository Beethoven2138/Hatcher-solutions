\documentclass{article}
\usepackage{graphicx} % Required for inserting images
\usepackage[english]{babel}
\usepackage{amssymb}
\usepackage{amsthm}
\usepackage{enumitem} 
\usepackage{amsmath}
\usepackage{amsfonts}
\usepackage{tikz}
\usetikzlibrary{matrix}
\usepackage[english]{babel}
\usepackage{mathtools}
\usepackage[a4paper, total={6in, 8in}]{geometry}
\usepackage{cite}


\newtheorem{theorem}{Theorem}[section]
\newtheorem{definition}[theorem]{Definition}
\newtheorem{lemma}[theorem]{Lemma}
\newtheorem{proposition}[theorem]{Proposition}
\newtheorem{corollary}[theorem]{Corollary}
\newtheorem{example}[theorem]{Example}
\newtheorem{remark}[theorem]{Remark}
\newtheorem{exercise}[theorem]{Exercise}


\title{Hatcher solutions}
\author{Saxon Supple}
\date{July 2025}

\begin{document}

\maketitle
\begin{exercise}
Construct an explicit deformation retraction of the torus with one point deleted onto a graph consisting of two circles intersecting in a point, namely, longitude and meridian circles of the torus.
\end{exercise}
\begin{proof}
We can model the torus as the disk $D^2:=\{(x,y)\in\mathbb{R}^2:x^2+y^2\leq 1\}$ where we split the boundary into $4$ equal arcs and identify opposite arcs under the quotient map $q:D^2\to T$. Let $A:=q(\partial D^2)$ be the subspace we wish to retract $T$ onto. Without loss of generality, suppose that we remove the point $(0,0)$. Given a point $x$, we want to send it to $\frac{x}{\|x\|}$ along the line segment from $x$ to $\frac{x}{\|x\|}$. We thus define the homotopy \[F:T\times I\to T:(q(x),t)\mapsto q\left(x(1-t)+\frac{x}{\|x\|}t\right).\] This homotopes between the identity map and a retraction of $T$ onto $A$ while fixing all points of $A$ throughout, and hence is a deformation retraction.
\end{proof}

\begin{exercise}
Construct an explicit deformation retraction of $\mathbb{R}^n\setminus\{0\}$ onto $S^{n-1}$.
\end{exercise}
\begin{proof}
We define a homotopy \[F:\mathbb{R}^n\setminus\{0\}\times I\to S^{n-1}:(x,t)\mapsto x(1-t)+\frac{x}{\|x\|}t.\] This homotopes between the identity map on $\mathbb{R}^n\setminus\{0\}$ and a retraction of $\mathbb{R}^n\setminus\{0\}$ onto $S^{n-1}$, and hence is a deformation retraction.
\end{proof}

\begin{exercise}
\begin{enumerate}
\item[(a)] Show that the composition of homotopy equivalences $X\to Y$ and $Y\to Z$ is a homotopy equivalence $X\to Z$. Deduce that homotopy equivalence is an equivalence relation.
\item[(b)] Show that the relation of homotopy among maps $X\to Y$ is an equivalence relation.
\item[(c)] Show that a map homotopic to a homotopy equivalence is a homotopy equivalence.
\end{enumerate}
\end{exercise}
\begin{proof}
\begin{enumerate}
\item[(a)] Let $\phi:X\to Y$ and $\psi: Y\to Z$ be homotopy equivalences. There then exist homotopy inverses $\phi_1:Y\to X$ and $\psi_1: Z\to Y$ of $\phi$ and $\psi$ respectively.\[(\phi_1\circ\psi_1)\circ(\psi\circ\phi)= \phi_1\circ(\psi_1\circ\psi)\circ\phi\simeq\phi_1\circ\phi\simeq\text{id}_X\] and\[(\psi\circ\phi)\circ(\phi_1\circ\psi_1)=\psi\circ(\phi\circ\phi_1)\circ\psi_1\simeq\psi\circ\psi_1\simeq\text{id}_Z\]so $\psi\circ\phi$ is a homotopy equivalence.
\item[(b)] Let $\phi,\psi,\rho$ be continuous maps from $X$ to $Y$. Clearly $\phi\simeq\phi$. Furthermore, if $F:X\times I\to Y$ is a homotopy from $\phi$ to $\psi$, then $G:X\times I\to Y:(x,t)\mapsto F(x,1-t)$ is a homotopy from $\psi$ to $\phi$. Finally, let $A:X\times I\to Y$ be a homotopy from $\phi$ to $\psi$ and let $B:X\times I\to Y$ be a homotopy from $\psi$ to $\rho$. We define a map $C:X\times I\to Y$ by \[C(x,t)=\begin{cases}
    A(x,2t)\text{ if }0\leq t\leq \frac{1}{2}\\B(x,2t-1)\text{ if }\frac{1}{2}<t\leq 1.
\end{cases}\]$X\times[0,\frac{1}{2}]$ and $X\times[\frac{1}{2},1]$ are both closed subsets of $X\times I$ with $X\times[0,\frac{1}{2}]\cup X\times[\frac{1}{2},1]=X\times I$. Furthermore, $C_{|{X\times[0,\frac{1}{2}]}}$ and $C_{|{X\times[\frac{1}{2},1]}}$ are both continuous with respect to the induced topologies on $X\times[0,\frac{1}{2}]$ and $X\times[\frac{1}{2},1]$ respectively. Hence, $C$ is continuous on $X\times I$ and is thus a homotopy from $\phi$ to $\rho$.
\item[(c)] Let $\phi:X\to Y$ be a homotopy equivalence with homotopy inverse $\psi:Y\to X$. Let $\phi_1:X\to Y$ be homotopic to $\phi$. Then \[\phi_1\circ\psi\simeq\phi\circ\psi\simeq\text{id}_Y\] and \[\psi\circ\phi_1\simeq\psi\circ\phi\simeq\text{id}_X.\] Hence $\phi_1$ is a homotopy equivalence.
\end{enumerate}
\end{proof}

\begin{exercise}
A $\textbf{deformation retraction in the weak sense}$ of a space $X$ to a subspace $A$ is a homotopy $f_t:X\to X$ such that $f_0=\mathbb{1}$, $f_1(X)\subset A$, and $f_t(A)\subset A$ for all $t$. Show that if $X$ deformation retracts to $A$ in this weak sense, then the inclusion $A\hookrightarrow X$ is a homotopy equivalence.
\end{exercise}
\begin{proof}
Let $\iota:A\to X$ denote the inclusion $A\hookrightarrow X$. Define $\phi:X\to A:x\mapsto f_1(x)$, which is well-defined since $f_1(X)\subset A$. Then \[\phi\circ\iota={f_1}_{|A}\simeq {f_0}_{|A}=\text{id}_A.\]This is a valid homotopy of maps $A\to A$ since $f_1(A)\subset A\forall t$. Furthermore,\[\iota\circ\phi=f_1\simeq\text{id}_X.\] Hence $\iota$ is a homotopy equivalence with homotopy inverse $\phi$.
\end{proof}

\begin{exercise}
Show that if a space $X$ deformation retracts to a point $x\in X$, then for each neighbourhood $U$ of $x$ in $X$ there exists a neighbourhood $V\subset U$ of $x$ such that the inclusion map $V\hookrightarrow U$ is nullhomotopic.
\end{exercise}
\begin{proof}
Without loss of generality, let $U$ be an open neighbourhood of $x$. Define \[F:X\times I\to X:(y,t)\mapsto f_t(y)\] and let $A:=F^{-1}(U)$. $A$ is an open set containing the slice $\{x\}\times I$, so by the Tube Lemma there exists a tube $V\times I\subset X\times I$, where $V$ is open and contains $x$. Furthermore, $V\times I\subset A$, so $V$ consists only of points in $U$ (since $V\times\{0\}\subset U\times\{0\}$) which stay in $U$ throughout the homotopy $f_t$. Let $\iota:V\hookrightarrow U$ be the inclusion map $V\hookrightarrow U$. Then $f_t:X\to X$ restricts to a homotopy ${f_t}_{|V}:V\to U$ and hence $\iota\simeq {f_1}_{|V}$, where ${f_1}_{|V}$ is the constant map, so $\iota$ is nullhomotopic.
\end{proof}

\end{document}
