\documentclass{article}
\usepackage{graphicx} % Required for inserting images
\usepackage[english]{babel}
\usepackage{amssymb}
\usepackage{amsthm}
\usepackage{enumitem} 
\usepackage{amsmath}
\usepackage{amsfonts}
\usepackage{tikz}
\usetikzlibrary{matrix}
\usepackage[english]{babel}
\usepackage{mathtools}
\usepackage[a4paper, total={6in, 8in}]{geometry}
\usepackage{cite}


\newtheorem{theorem}{Theorem}[section]
\newtheorem{definition}[theorem]{Definition}
\newtheorem{lemma}[theorem]{Lemma}
\newtheorem{proposition}[theorem]{Proposition}
\newtheorem{corollary}[theorem]{Corollary}
\newtheorem{example}[theorem]{Example}
\newtheorem{remark}[theorem]{Remark}
\newtheorem{exercise}[theorem]{Exercise}


\title{Hatcher solutions}
\author{Saxon Supple}
\date{July 2025}

\begin{document}

\maketitle
\begin{exercise}
Construct an explicit deformation retraction of the torus with one point deleted onto a graph consisting of two circles intersecting in a point, namely, longitude and meridian circles of the torus.
\end{exercise}
\begin{proof}
We can model the torus as the disk $D^2:=\{(x,y)\in\mathbb{R}^2:x^2+y^2\leq 1\}$ where we split the boundary into $4$ equal arcs and identify opposite arcs under the quotient map $q:D^2\to T$. Let $A:=q(\partial D^2)$ be the subspace we wish to retract $T$ onto. Without loss of generality, suppose that we remove the point $(0,0)$. Given a point $x$, we want to send it to $\frac{x}{\|x\|}$ along the line segment from $x$ to $\frac{x}{\|x\|}$. We thus define the homotopy \[F:T\times I\to T:(q(x),t)\mapsto q\left(x(1-t)+\frac{x}{\|x\|}t\right).\] This homotopes between the identity map and a retraction of $T$ onto $A$ while fixing all points of $A$ throughout, and hence is a deformation retraction.
\end{proof}

\begin{exercise}
Construct an explicit deformation retraction of $\mathbb{R}^n\setminus\{0\}$ onto $S^{n-1}$.
\end{exercise}
\begin{proof}
We define a homotopy \[F:\mathbb{R}^n\setminus\{0\}\times I\to S^{n-1}:(x,t)\mapsto x(1-t)+\frac{x}{\|x\|}t.\] This homotopes between the identity map on $\mathbb{R}^n\setminus\{0\}$ and a retraction of $\mathbb{R}^n\setminus\{0\}$ onto $S^{n-1}$, and hence is a deformation retraction.
\end{proof}

\end{document}
